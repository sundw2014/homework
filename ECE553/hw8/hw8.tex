\documentclass[11pt]{report}
% this template is originally from Roy Dong's ECE 515.
% Edited by Dawei Sun
%%%%%%%%%%%%%%%%%%%%%%%%%%%%%%%%%%%%%%%%%%%%%%%%%%%%%%%%%%%%%%%%%%
% Set the margins of our document.
\usepackage[margin = 1 in]{geometry}

%%%%%%%%%%%%%%%%%%%%%%%%%%%%%%%%%%%%%%%%%%%%%%%%%%%%%%%%%%%%%%%%%%
% Import commands for custom header.
\usepackage{fancyhdr}
\pagestyle{fancy}

%%%%%%%%%%%%%%%%%%%%%%%%%%%%%%%%%%%%%%%%%%%%%%%%%%%%%%%%%%%%%%%%%%
% Allow ourselves to do equations!
\usepackage{amsmath,amssymb,amsthm,amsfonts}
\usepackage{upgreek}
\usepackage{mathtools}
\usepackage{bbm}
\usepackage{hyperref}

%%%%%%%%%%%%%%%%%%%%%%%%%%%%%%%%%%%%%%%%%%%%%%%%%%%%%%%%%%%%%%%%%%
% Nicer formatting for enumerate commands.
\usepackage[shortlabels]{enumitem}

\usepackage{algorithm2e}
\usepackage[noend]{algpseudocode}

%%%%%%%%%%%%%%%%%%%%%%%%%%%%%%%%%%%%%%%%%%%%%%%%%%%%%%%%%%%%%%%%%%
% Colored text and include images.
\usepackage{color}
\usepackage[dvipsnames]{xcolor}
\usepackage{graphicx}

\usepackage{listings}
\usepackage{multicol}
%%%%%%%%%%%%%%%%%%%%%%%%%%%%%%%%%%%%%%%%%%%%%%%%%%%%%%%%%%%%%%%%%%
% Some custom macros to make life easier.
\newcommand{\mc}{\mathcal}
\newcommand{\mb}{\mathbb}
\newcommand{\reals}{\mathbb{R}}

\newcommand{\T}{\intercal}
\newcommand{\E}[1]{\mathbb{E}\left[#1\right]}

%%%%%%%%%%%%%%%%%%%%%%%%%%%%%%%%%%%%%%%%%%%%%%%%%%%%%%%%%%%%%%%%%%
%%%%%%%%%%%%%%%%%%%%%%%%%%%%%%%%%%%%%%%%%%%%%%%%%%%%%%%%%%%%%%%%%%
%%%%%%%%%%%%%%%%%%%%%%%%%%%%%%%%%%%%%%%%%%%%%%%%%%%%%%%%%%%%%%%%%%

\lhead{ECE 553 - Fall 2020 at University of Illinois at Urbana-Champaign}
\rhead{\textcolor{red}{Dawei Sun (daweis2)}}

%%%%%%%%%%%%%%%%%%%%%%%%%%%%%%%%%%%%%%%%%%%%%%%%%%%%%%%%%%%%%%%%%%
%%%%%%%%%%%%%%%%%%%%%%%%%%%%%%%%%%%%%%%%%%%%%%%%%%%%%%%%%%%%%%%%%%
%%%%%%%%%%%%%%%%%%%%%%%%%%%%%%%%%%%%%%%%%%%%%%%%%%%%%%%%%%%%%%%%%%

\begin{document}

%%%%%%%%%%%%%%%%%%%%%%%%%%%%%%%%%%%%%%%%%%%%%%%%%%%%%%%%%%%%%%%%%%
%%%%%%%%%%%%%%%%%%%%%%%%%%%%%%%%%%%%%%%%%%%%%%%%%%%%%%%%%%%%%%%%%%
%%%%%%%%%%%%%%%%%%%%%%%%%%%%%%%%%%%%%%%%%%%%%%%%%%%%%%%%%%%%%%%%%%

\section*{Exercise 6.1}
First, $P(t_1) = -\frac{1}{2}Y(t_1)X(t_1)^{-1} = M$.

\noindent Second, using the fact that $\frac{d}{dt}{X^{-1}(t)} = -X^{-1}(t) \dot{X}(t) X^{-1}(t)$, we have
\begin{align*}
\dot{P}(t) = &-\frac{1}{2}\left(\dot{Y}(t)X^{-1}(t) - X^{-1}(t) \dot{X}(t) X^{-1}(t)\right)\\
= & -\frac{1}{2}\left(\left(2Q(t)X(t) - A^\T(t)Y(t)\right)X^{-1}(t) - Y(t)X^{-1}(t)\left(A(t)X(t) + \frac{1}{2}B(t)R^{-1}(t)B^\T(t)Y(t)\right)X^{-1}(t)\right)\\
= & - Q(t) - A^\T(t)P(t) - P(t)A(t) + P(t)B(t)R^{-1}(t)B^\T(t)P(t).
\end{align*}

\section*{Exercise 6.2}
(1) Obviously, if $P(t)$ satisfies (6.14), then $P^\T(t)$ also satisfies (6.14):
\begin{align*}
\frac{d}{dt}P^\T(t) = & \left(\frac{d}{dt}P(t)\right)^\T = \left(- A^\T(t)P(t) - P(t)A(t) - Q(t) + P(t)B(t)R^{-1}(t)B^\T(t)P(t)\right)^\T\\ = & - A^\T(t)P^\T(t) - P^\T(t)A(t) - Q(t) + P^\T(t)B(t)R^{-1}(t)B^\T(t)P^\T(t).
\end{align*}
Together with fact that the boundary condition $P(t_1) = M$ is symmetric, $P(t)$ is symmetric for all $t$.

\noindent (2) First, rewrite the ODE as follows
\begin{align*}
\dot{P}(t) = & - A^\T(t)P(t) - P(t)A(t) - Q(t) + P(t)B(t)R^{-1}(t)B^\T(t)P(t)\\ = & - \left(A(t) - \frac{1}{2}B(t)R^{-1}(t)B^\T(t)P(t)\right)^\T P(t) - P(t)\left(A(t) - \frac{1}{2}B(t)R^{-1}(t)B^\T(t)P(t)\right) - Q(t)\\ := &- S(t)^\T P(t) - P(t)S(t) - Q(t).
\end{align*}
It is easy to verify that if there exists a solution $P(t)$, then it satisfies
\begin{align*}
P(t) = & \Phi(t, t_1) P(t_1) \Phi(t, t_1)^\T + \int_{t_1}^t \Phi(t, \tau) (-Q(t)) \Phi(t, \tau)^\T d \tau\\ = &\Phi(t, t_1) P(t_1) \Phi(t, t_1)^\T + \int_{t}^{t_1} \Phi(t, \tau) Q(t) \Phi(t, \tau)^\T d \tau,
\end{align*}
where $\Phi(t,\tau)$ is the solution of
$$\partial_t \Phi(t, \tau) = -S(t) \Phi(t, \tau),~\Phi(\tau, \tau) = I.$$ Since $\Phi(t,\tau)$ is non-singular, and $P(t_1) = M \succeq 0$ and $Q(t) \succeq 0$, $P(t)$ must be positive semi-definite.

\noindent (3) As shown in (2), if $M \succ 0$ or $Q(t) \succ 0$ on a measurable subset of $[t, t_1]$, then $P(t) \succ 0$.

\section*{Exercise 6.3}
Let us rewrite the cost function such that everything outside the integral is independent of $u$. Let $P(t)$ be the solution of (6.14).
\begin{align*}
J(u) = & \int_{t_0}^{t_1} \left(x^\T(t) Q(t) x(t) + u^\T(t) R(t) u(t)\right) d t + x^\T(t_1)P(t_1)x(t_1) \\ = & \int_{t_0}^{t_1} \left(x^\T Q(t) x(t) + u^\T(t) R(t) u(t)\right) d t + x^\T(t_0)P(t_0)x(t_0) + \int_{t_0}^{t_1} \frac{d}{dt}\left(x^\T(t)P(t)x(t)\right) dt \\ = & \int_{t_0}^{t_1} \left(u^\T(t) R(t) u(t) + 2u^\T(t)B^\T(t)P(t)x(t) + x^\T(t)P(t)B(t)R^{-1}(t)B^\T(t)P(t)x(t)\right) d t + x^\T(t_0)P(t_0)x(t_0) \\ = & \int_{t_0}^{t_1} \left(u(t) + R^{-1}(t)B^\T(t)P(t)x(t)\right)^\T R(t)\left(u(t) + R^{-1}(t)B^\T(t)P(t)x(t)\right) d t + x^\T(t_0)P(t_0)x(t_0).
\end{align*}
Obviously, the unique global minimizer is $u^*(t) = - R^{-1}(t)B^\T(t)P(t)x(t)$.

\section*{Exercise 6.4}
In this case, $A = \begin{bmatrix}0 & 1\\ 0 & 0\end{bmatrix}$, $B = \begin{bmatrix}0\\1\end{bmatrix}$, $Q = I$, and $R = 1$. Solve the ARE, and we get $P = \begin{bmatrix}\sqrt{3} & 1\\ 1 & \sqrt{3}\end{bmatrix}$.

\noindent Furthermore, the Hamiltonian matrix $\mathcal{H}(t) = \begin{bmatrix}0 & 1 & 0 & 0\\ 0 & 0 & 0 & \frac{1}{2}\\ 2 & 0 & 0 & 0 \\0 & 2 & -1 & 0\end{bmatrix}$. Using the results in Exercise 6.1, we have that $P(t) = -\frac{1}{2}Y(t)X^{-1}(t)$, where $X$ and $Y$ satisfy
\[
\begin{bmatrix}\dot{X}\\\dot{Y}\end{bmatrix} = \mathcal{H}\begin{bmatrix}X\\Y\end{bmatrix},~X(t_1) = I, Y(t_1) = -2M.
\]
Moreover, solve this ODE, and we get
\[
\begin{bmatrix}X(t)\\Y(t)\end{bmatrix} = e^{\mathcal{H}(t-t_1)} \begin{bmatrix}I\\-2M\end{bmatrix}.
\]
The Hamiltonian matrix $\mathcal{H}$ is diagonalizable:
\[
\mathcal{H} = \begin{bmatrix}P_{11} & P_{12}\\P_{21} & P_{22}\end{bmatrix} \begin{bmatrix}\lambda_1 & 0 & 0 & 0\\ 0 & \lambda_2 & 0 & 0\\ 0 & 0 & \lambda_3 & 0\\0 & 0 & 0 & \lambda_4\end{bmatrix} \begin{bmatrix}\tilde{P}_{11} & \tilde{P}_{12}\\\tilde{P}_{21} & \tilde{P}_{22}\end{bmatrix},
\]
where $Re(\lambda_1) = Re(\lambda_2) = -Re(\lambda_3) = -Re(\lambda_4) > 0$.
For large $t_1$, we have that
\begin{align*}
e^{\mathcal{H}(t-t_1)} = & \begin{bmatrix}P_{11} & P_{12}\\P_{21} & P_{22}\end{bmatrix} \begin{bmatrix}e^{\lambda_1(t-t_1)} & 0 & 0 & 0\\ 0 & e^{\lambda_2(t-t_1)} & 0 & 0\\ 0 & 0 & e^{\lambda_3(t-t_1)} & 0\\0 & 0 & 0 & e^{\lambda_4(t-t_1)}\end{bmatrix} \begin{bmatrix}\tilde{P}_{11} & \tilde{P}_{12}\\\tilde{P}_{21} & \tilde{P}_{22}\end{bmatrix}\\ \approx & \begin{bmatrix}P_{11} & P_{12}\\P_{21} & P_{22}\end{bmatrix} \begin{bmatrix}0 & 0 & 0 & 0\\ 0 & 0 & 0 & 0\\ 0 & 0 & e^{\lambda_3(t-t_1)} & 0\\0 & 0 & 0 & e^{\lambda_4(t-t_1)}\end{bmatrix} \begin{bmatrix}\tilde{P}_{11} & \tilde{P}_{12}\\\tilde{P}_{21} & \tilde{P}_{22}\end{bmatrix}.
\end{align*}
Let $\Lambda = \begin{bmatrix}e^{\lambda_3(t-t_1)} & 0\\0 & e^{\lambda_4(t-t_1)}\end{bmatrix}$. Then, we have that
\[
\begin{bmatrix}X(t)\\Y(t)\end{bmatrix} \approx \begin{bmatrix}P_{12}\Lambda\tilde{P}_{21} & P_{12}\Lambda\tilde{P}_{22}\\P_{22}\Lambda\tilde{P}_{21} & P_{22}\Lambda\tilde{P}_{22}\end{bmatrix} \begin{bmatrix}I\\-2M\end{bmatrix} = \begin{bmatrix}P_{12}(\Lambda\tilde{P}_{21}-2P_{12}\Lambda\tilde{P}_{22}M)\\P_{22}(\Lambda\tilde{P}_{21}-2P_{12}\Lambda\tilde{P}_{22}M)\end{bmatrix}.
\]
Thus, $P(t) \to -\frac{1}{2}Y(t)X^{-1}(t) = -\frac{1}{2}P_{22}P_{12}^{-1} = \begin{bmatrix}\sqrt{3} & 1\\ 1 & \sqrt{3}\end{bmatrix}$ as $t_1 \to \infty$, which is independent of $M$.

\section*{Exercise 6.5}
All the claims except $P \succ 0$ hold. There are three steps in the proof where controllability or observability is used.

\paragraph{Boundedness of the optimal cost. (p. 154)} Stablizability guarantees that there exists $K$ such that all the eigenvalues of the closed-loop dynamics $(A+BK)$ have negative real parts. Controlled by the controller $u = Kx$, $x^\T(t)x(t) \to 0$. Since $0 \leq x^\T(t)Qx(t) + u^\T(t)Ru(t) = x^\T(t)(Q+K^\T K)x(t) \leq \|Q+K^\T K\|_2 x^\T(t)x(t)$, the running cost also converges to $0$ exponentially, and thus $J(u)$ is bounded. Optimal cost is smaller than $J(u)$ and hence also bounded.

\paragraph{Closed-loop stability. (p. 157)} It is also true that if the output and the input of a detectable linear time-invariant system converge to $0$, then so does the state. One way to see this is as follows:
by detectability, there exists a matrix $L$ such that $(A - LC)$ has eigenvalues with negative real parts; rewriting the system dynamics as $\dot{x} = (A - LC)x + Ly + Bu$, it is easy to show that $x \to 0$ when $y, u \to 0$.

\paragraph{$P \succ 0$. (p. 158)} We know that the derivatives of the output at $t=0$ are
\[
\left[\begin{array}{c}
\mathbf{y}\left(t_{0}\right) \\
\dot{\mathbf{y}}\left(t_{0}\right) \\
\ddot{\mathbf{y}}\left(t_{0}\right) \\
\vdots \\
\mathbf{y}^{(n-1)}\left(t_{0}\right)
\end{array}\right]^{(n p) \times 1}=\left[\begin{array}{c}
\mathbf{C} \\
\mathbf{C A} \\
\mathbf{C A}^{2} \\
\vdots \\
\mathbf{C A}^{n-1}
\end{array}\right]^{(n p) \times n} \quad \times \mathbf{x}\left(t_{0}\right)=\mathcal{O} \mathbf{x}\left(t_{0}\right).
\]
Since $\mathcal{O}$ is not full-rank, there exists $x(t_0) \neq 0$ such that $\mathcal{O}x(t_0) = 0$. Starting from such an $x(t_0)$ and setting $u \equiv 0$, we have that $y(t) = Cx(t) = Ce^{At}x(t_0) = C\left(\sum_{i=0}^{\infty}\alpha_i A^i\right)x(t_0) \equiv 0$. Thus, $J(u) = 0$ and $u \equiv 0$ is clearly the optimal control. Thus, the correct claim is that ``$P$ is the unique symmetric positive \textbf{semi}-definite solution of the ARE". The proof of the uniqueness still holds since it is for ``the class of positive semidefinite matrices".

\section*{Exercise 6.6}
The ARE is $-\frac{b^2}{r}p^2 + 2ap + q = 0$. Solve it, and we get $b^2\frac{p}{r} = a + \sqrt{a^2+\frac{qb^2}{r}}$. The eigenvalue of the closed-loop system is $\lambda = a - b^2\frac{p}{r}$. Thus, when $r \to 0$, $\lambda \to -\infty$. When $r \to \infty$, $\lambda \to -a$.
\end{document}
