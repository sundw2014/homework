\documentclass[11pt]{report}

%%%%%%%%%%%%%%%%%%%%%%%%%%%%%%%%%%%%%%%%%%%%%%%%%%%%%%%%%%%%%%%%%%
% Set the margins of our document.
\usepackage[margin = 1 in]{geometry}

%%%%%%%%%%%%%%%%%%%%%%%%%%%%%%%%%%%%%%%%%%%%%%%%%%%%%%%%%%%%%%%%%%
% Import commands for custom header.
\usepackage{fancyhdr}
\pagestyle{fancy}

%%%%%%%%%%%%%%%%%%%%%%%%%%%%%%%%%%%%%%%%%%%%%%%%%%%%%%%%%%%%%%%%%%
% Allow ourselves to do equations!
\usepackage{amsmath,amssymb,amsthm}

%%%%%%%%%%%%%%%%%%%%%%%%%%%%%%%%%%%%%%%%%%%%%%%%%%%%%%%%%%%%%%%%%%
% Nicer formatting for enumerate commands.
\usepackage[shortlabels]{enumitem}

%%%%%%%%%%%%%%%%%%%%%%%%%%%%%%%%%%%%%%%%%%%%%%%%%%%%%%%%%%%%%%%%%%
% Colored text and include images.
\usepackage{color}
\usepackage{graphicx}

%%%%%%%%%%%%%%%%%%%%%%%%%%%%%%%%%%%%%%%%%%%%%%%%%%%%%%%%%%%%%%%%%%
% Some custom macros to make life easier.
\newcommand{\mc}{\mathcal}
\newcommand{\mb}{\mathbb}

\newcommand{\T}{\intercal}

%%%%%%%%%%%%%%%%%%%%%%%%%%%%%%%%%%%%%%%%%%%%%%%%%%%%%%%%%%%%%%%%%%
%%%%%%%%%%%%%%%%%%%%%%%%%%%%%%%%%%%%%%%%%%%%%%%%%%%%%%%%%%%%%%%%%%
%%%%%%%%%%%%%%%%%%%%%%%%%%%%%%%%%%%%%%%%%%%%%%%%%%%%%%%%%%%%%%%%%%

\lhead{ECE 515 - Fall 2019 at University of Illinois at Urbana-Champaign}
\rhead{\textcolor{red}{HW8 - Template}}
\lfoot{Submitted by: \textcolor{red}{\textit{daweis2}}}
\rfoot{Due: Thursday, November 14 at 2pm}

%%%%%%%%%%%%%%%%%%%%%%%%%%%%%%%%%%%%%%%%%%%%%%%%%%%%%%%%%%%%%%%%%%
%%%%%%%%%%%%%%%%%%%%%%%%%%%%%%%%%%%%%%%%%%%%%%%%%%%%%%%%%%%%%%%%%%
%%%%%%%%%%%%%%%%%%%%%%%%%%%%%%%%%%%%%%%%%%%%%%%%%%%%%%%%%%%%%%%%%%

\begin{document}

%%%%%%%%%%%%%%%%%%%%%%%%%%%%%%%%%%%%%%%%%%%%%%%%%%%%%%%%%%%%%%%%%%
%%%%%%%%%%%%%%%%%%%%%%%%%%%%%%%%%%%%%%%%%%%%%%%%%%%%%%%%%%%%%%%%%%
%%%%%%%%%%%%%%%%%%%%%%%%%%%%%%%%%%%%%%%%%%%%%%%%%%%%%%%%%%%%%%%%%%

%\pagebreak
\section*{Problem 1}

Suppose we have an LTI system with an input disturbance:
\[
\dot x = Ax + B(u+w)
\]
\[
y = Cx
\]
If we knew $w(t)$, we could control the system with $u(t) = -w(t) - K \widehat x(t)$, where $\widehat x(t)$ is our state estimate from some observer.

However, we don't know $w(t)$. Suppose that we know the dynamics of $w(t)$:
\[
\dot z = A_m z
\]
\[
w = C_m z
\]
Thus, we know $A_m, C_m$, but not the value of $z$ or $w$.
Let's augment our state-space model as follows:
\[
\begin{bmatrix}
\dot x \\
\dot z
\end{bmatrix} =
\begin{bmatrix}
A & B C_m \\
0 & A_m
\end{bmatrix}
\begin{bmatrix}
x \\
z
\end{bmatrix} +
\begin{bmatrix}
B \\
0
\end{bmatrix} u
\]
\[
y =
\begin{bmatrix}
C & 0
\end{bmatrix} x
\]
\begin{itemize}
\item
Is the new system controllable?
\item
Derive a full-state observer for this augmented system. Leave the observer gain as $L = \begin{bmatrix}L_1 \\ L_2 \end{bmatrix}$.
\item
Use the state estimate from this observer for a control law $u = -C_m \widehat z - K \widehat x$. (Note that our estimate of the unknown noise term $w$ is $C_m \widehat z$.) Write out the dynamics of the controller and observer, viewing $y$ as the input to this system and $u$ as the output. What are the eigenvalues of these `controller and observer' dynamics? (Again, leave the controller gain as $K$.)
\item
Write out the closed-loop dynamics for the controller, observer, and original system, using the states $(x, z, \widehat x, \widehat z)$.
\end{itemize}

%%%%%%%%%%%%%%%%%%%%%%%%%%%%%%%%%%%%%%%%%%%%%%%%%%%%%%%%%%%%%%%%%%

\subsection*{Solution}
\begin{itemize}
\item The input $u$ doesn't effect $z$ directly, nor indirectly via $x$, thus it is not controllable. Formally, choose $s$ as an eigenvalue of $A_m$, then last $m$ rows of $\begin{bmatrix} sI-A & -BC_m & B\\0 & sI-A_m & 0 \end{bmatrix}$ is not linearly independent, and thus the rank is less than $n+m$.
\item $$\begin{bmatrix} \dot{\hat{x}}\\ \dot{\hat{z}} \end{bmatrix} = \begin{bmatrix} A & BC_m \\0 & A_m \end{bmatrix} \begin{bmatrix} \hat{x}\\ \hat{z} \end{bmatrix} + \begin{bmatrix} B\\ 0 \end{bmatrix}u + \begin{bmatrix} L_1\\ L_2 \end{bmatrix} (y - C\hat{x})$$
\item $$\begin{bmatrix} \dot{\hat{x}}\\ \dot{\hat{z}} \end{bmatrix} = \begin{bmatrix} A-BK-L_1C & 0 \\-L_2C & A_m \end{bmatrix} \begin{bmatrix} \hat{x}\\ \hat{z} \end{bmatrix} + \begin{bmatrix} L_1\\ L_2 \end{bmatrix} y$$
$$u = \begin{bmatrix} -K & -C_m \end{bmatrix} \begin{bmatrix} \hat{x}\\ \hat{z} \end{bmatrix}$$
According to a conclusion from HW2, we know the eigevalues of $\begin{bmatrix} A-BK-L_1C & 0 \\-L_2C & A_m \end{bmatrix}$ is the union of eigenvalues of $A-BK-L_1C$ and $A_m$.
\item $$\begin{bmatrix} \dot{x} \\ \dot{z} \\ \dot{\hat{x}}\\ \dot{\hat{z}} \end{bmatrix} = \begin{bmatrix} A & BC_m & -BK & -BC_m\\ 0 & A_m & 0 & 0\\ L_1C & 0 & A-BK-L_1C & 0 \\ L_2C & 0 & -L_2C & A_m \end{bmatrix} \begin{bmatrix} x \\ z \\ \hat{x}\\ \hat{z} \end{bmatrix}$$
\end{itemize}
%%%%%%%%%%%%%%%%%%%%%%%%%%%%%%%%%%%%%%%%%%%%%%%%%%%%%%%%%%%%%%%%%%
%%%%%%%%%%%%%%%%%%%%%%%%%%%%%%%%%%%%%%%%%%%%%%%%%%%%%%%%%%%%%%%%%%
%%%%%%%%%%%%%%%%%%%%%%%%%%%%%%%%%%%%%%%%%%%%%%%%%%%%%%%%%%%%%%%%%%

\pagebreak
\section*{Problem 2}

Suppose we have the same setup as the previous problem, with system dynamics given by
\[
\dot x = Ax + B(u+w)
\]
\[
y = Cx
\]
and noise dynamics given by
\[
\dot z = A_m z
\]
\[
w = C_m z
\]
Rather than explicitly estimate the unknown state, let's filter the output. Let's add the following states:
\[
\dot x_a = A_m^\T x_a + C_m^\T y
\]
\begin{itemize}
\item
Derive an observer for $\widehat x$, ignoring the $w$ term.
%using $C_m \widehat z$ as a substitute for $w$. (We assume we can directly observe $\widehat z$ and do not need to estimate it.)
Again, leave $L$ in the equations without explicitly calculating these values.
\item
Stabilize the system with $u = -K_1 \widehat x - K_2 x_a$. Derive the dynamics for your controller and filter $(\widehat x,x_a)$, viewing $y$ as an input and $u$ as the output.
\item
What are the eigenvalues of your `controller and filter' dynamics?
\end{itemize}

%%%%%%%%%%%%%%%%%%%%%%%%%%%%%%%%%%%%%%%%%%%%%%%%%%%%%%%%%%%%%%%%%%

\subsection*{Solution}
\begin{itemize}
\item $\dot{\hat{x}} = A \hat{x} + Bu + L(y - C\hat{x})$
\item $$\begin{bmatrix} \dot{\hat{x}}\\ \dot{x}_a \end{bmatrix} = \begin{bmatrix} A-BK_1-LC & -BK_2 \\0 & A_m^\T \end{bmatrix} \begin{bmatrix} \hat{x}\\ x_a \end{bmatrix} + \begin{bmatrix} L\\ C_m^\T \end{bmatrix} y$$
$$u = \begin{bmatrix} -K_1 & -K_2 \end{bmatrix} \begin{bmatrix} \hat{x}\\ x_a \end{bmatrix}$$
\item According to a conclusion from HW2, we know the eigevalues of $\begin{bmatrix} A-BK_1-LC & -BK_2 \\0 & A_m^\T \end{bmatrix}$ is the union of eigenvalues of $A-BK_1-LC$ and $A_m^\T$.
\end{itemize}

%%%%%%%%%%%%%%%%%%%%%%%%%%%%%%%%%%%%%%%%%%%%%%%%%%%%%%%%%%%%%%%%%%
%%%%%%%%%%%%%%%%%%%%%%%%%%%%%%%%%%%%%%%%%%%%%%%%%%%%%%%%%%%%%%%%%%
%%%%%%%%%%%%%%%%%%%%%%%%%%%%%%%%%%%%%%%%%%%%%%%%%%%%%%%%%%%%%%%%%%

\end{document}
