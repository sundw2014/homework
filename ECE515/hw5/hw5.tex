\documentclass[11pt]{report}

%%%%%%%%%%%%%%%%%%%%%%%%%%%%%%%%%%%%%%%%%%%%%%%%%%%%%%%%%%%%%%%%%%
% Set the margins of our document.
\usepackage[margin = 1 in]{geometry}

%%%%%%%%%%%%%%%%%%%%%%%%%%%%%%%%%%%%%%%%%%%%%%%%%%%%%%%%%%%%%%%%%%
% Import commands for custom header.
\usepackage{fancyhdr}
\pagestyle{fancy}

%%%%%%%%%%%%%%%%%%%%%%%%%%%%%%%%%%%%%%%%%%%%%%%%%%%%%%%%%%%%%%%%%%
% Allow ourselves to do equations!
\usepackage{amsmath,amssymb,amsthm}

%%%%%%%%%%%%%%%%%%%%%%%%%%%%%%%%%%%%%%%%%%%%%%%%%%%%%%%%%%%%%%%%%%
% Nicer formatting for enumerate commands.
\usepackage[shortlabels]{enumitem}

%%%%%%%%%%%%%%%%%%%%%%%%%%%%%%%%%%%%%%%%%%%%%%%%%%%%%%%%%%%%%%%%%%
% Colored text and include images.
\usepackage{color}
\usepackage{graphicx}

\usepackage{upgreek}

%%%%%%%%%%%%%%%%%%%%%%%%%%%%%%%%%%%%%%%%%%%%%%%%%%%%%%%%%%%%%%%%%%
% Some custom macros to make life easier.
\newcommand{\mc}{\mathcal}
\newcommand{\mb}{\mathbb}

%%%%%%%%%%%%%%%%%%%%%%%%%%%%%%%%%%%%%%%%%%%%%%%%%%%%%%%%%%%%%%%%%%
%%%%%%%%%%%%%%%%%%%%%%%%%%%%%%%%%%%%%%%%%%%%%%%%%%%%%%%%%%%%%%%%%%
%%%%%%%%%%%%%%%%%%%%%%%%%%%%%%%%%%%%%%%%%%%%%%%%%%%%%%%%%%%%%%%%%%

\lhead{ECE 515 - Fall 2019 at University of Illinois at Urbana-Champaign}
\rhead{\textcolor{red}{HW5 - Template}}
\lfoot{Submitted by: \textcolor{red}{\textit{daweis2}}}
\rfoot{Due: Thursday, October 17 at 2pm}

%%%%%%%%%%%%%%%%%%%%%%%%%%%%%%%%%%%%%%%%%%%%%%%%%%%%%%%%%%%%%%%%%%
%%%%%%%%%%%%%%%%%%%%%%%%%%%%%%%%%%%%%%%%%%%%%%%%%%%%%%%%%%%%%%%%%%
%%%%%%%%%%%%%%%%%%%%%%%%%%%%%%%%%%%%%%%%%%%%%%%%%%%%%%%%%%%%%%%%%%

\begin{document}

%%%%%%%%%%%%%%%%%%%%%%%%%%%%%%%%%%%%%%%%%%%%%%%%%%%%%%%%%%%%%%%%%%
%%%%%%%%%%%%%%%%%%%%%%%%%%%%%%%%%%%%%%%%%%%%%%%%%%%%%%%%%%%%%%%%%%
%%%%%%%%%%%%%%%%%%%%%%%%%%%%%%%%%%%%%%%%%%%%%%%%%%%%%%%%%%%%%%%%%%

%%%%%%%%%%%%%%%%%%%%%%%%%%%%%%%%%%%%%%%%%%%%%%%%%%%%%%%%%%%%%%%%%%
%%%%%%%%%%%%%%%%%%%%%%%%%%%%%%%%%%%%%%%%%%%%%%%%%%%%%%%%%%%%%%%%%%
%%%%%%%%%%%%%%%%%%%%%%%%%%%%%%%%%%%%%%%%%%%%%%%%%%%%%%%%%%%%%%%%%%
\section*{Problem 1}

Consider the following LTI model:
\[
\dot x =
\begin{bmatrix}
-7 & -2 & 6 \\
2 & -3 & -2 \\
-2 & -2 & 1
\end{bmatrix}
x +
\begin{bmatrix}
1 & 1 \\
1 & -1 \\
1 & 0
\end{bmatrix}
u
\]
\[
y =
\begin{bmatrix}
-1 & -1 & 2 \\
1 & 1 & -1
\end{bmatrix}
x
\]

Check controllability using:
\begin{enumerate}[label=\alph*)]
\item
the controllability matrix.
\item
the rows of $\overline B = M^{-1} B$, where $M$ is chosen such that $M^{-1} A M$ is diagonal.
\item
the Hautus-Rosenbrock condition.
\end{enumerate}

You may use a calculator, MATLAB, or another computational device, but be sure you would know how to do this manually if needed.

%%%%%%%%%%%%%%%%%%%%%%%%%%%%%%%%%%%%%%%%%%%%%%%%%%%%%%%%%%%%%%%%%%

\subsection*{Solution}
\begin{enumerate}[label=\alph*)]
\item
We have $C = [B | AB | A^2B]$, and $C = \begin{bmatrix} 1 & 1 & -3 & -5 & 9 & 25\\  1 & -1 & -3 & 5 & 9 & -25\\  1 & 0 & -3 & 0 & 9 & 0 \end{bmatrix}$, and thus $\text{rank}(C) = 2$. So, the model is uncontrollable.
\item
We have $M = \begin{bmatrix} 0.707 & 0.707 & -0.577\\ 0 & -0.707 & -0.577\\ 0.707 & 0 & -0.577\end{bmatrix}$, and $\overline B = M^{-1} B = \begin{bmatrix} 0 & 0\\ 0 & 1.414\\ -1.732 & 0\end{bmatrix}$. There is a $0$-row in $\overline B$, so the model is uncontrollable.
\item
The eigen values of $A$ are $s = [-1, -5, -3]$. For $s_1 = -1$, we have $\text{rank}(-I - A|B) = 2$, and thus the model is uncontrollable.

\end{enumerate}
%%%%%%%%%%%%%%%%%%%%%%%%%%%%%%%%%%%%%%%%%%%%%%%%%%%%%%%%%%%%%%%%%%

\pagebreak
\section*{Problem 2}

Let $A$ be an $n \times n$ matrix and $B$ be an $n \times r$ matrix, both with real entries. Suppose $(A,B)$ is controllable. Prove or disprove the following statements. (If the statement is false, then producing a counterexample will suffice.)

\begin{enumerate}[label=\alph*)]
\item The pair $(A^2,B)$ is controllable.
\item Let $k(\cdot)$ be a known $n$-dimensional function, piecewise continuous in $t$. Consider the model:
\[
\dot x = Ax + Bu + k(t)
\]
This model is controllable, in the sense that for any initial state $x_0$ and any target final state $x_f$, there exists a control $u(\cdot)$ that directs the system from $x_0$ to $x_f$ in finite time.
\item Given that the model $\dot x = Ax + Bu$ has the initial condition $x(0) = x_0 \neq 0$, it is possible to find a piecewise continuous control, defined on $[0,\infty)$, such that the model is brought to rest at $t = 1$, i.e. $x(t) = 0$ for all $t \ge 1$.
\item Suppose the model starts at $x(0) = 0$; there exists a piecewise continuous control which will bring the state to $x_f \in \mb{R}^n$ by time $t \ge 1$ and maintain that value $x(t) = x_f$ for all $t \ge 1$.
\end{enumerate}

%%%%%%%%%%%%%%%%%%%%%%%%%%%%%%%%%%%%%%%%%%%%%%%%%%%%%%%%%%%%%%%%%%

\subsection*{Solution}
\begin{enumerate}[label=\alph*)]
\item
False. Let $A = \begin{bmatrix} 0&1\\1&0 \end{bmatrix}$, then $A^2 = I$. Let $B=[1,3]^T$, then $\text{rank}([B, AB]) = 2$, $\text{rank}([B, A^2B]) = 1$.
\item
True. Decompose $x$ into two components, $x = x_1 + x_2$, and $\dot{x_1} = A x_1 + B u$, $\dot{x_2} = A x_2 + k(t)$. Because $(A, B)$ is controllable, there exists $t_f$, such that for any $x_0$ and $x_f$ we can find a control $u(t)$. Fix this $t_f$ and set the initial state of $x_2$ to $0$, i.e. $x_2(0) = 0$, then $x_2(t_f)$ is fixed. Then, let $x_1(0) = x_0$, we can find a $u(t)$ such that $x_1(t_f) = x_f - x_2(t_f)$. This $u(t)$ will dirve the original model from $x_0$ to $x_f$.
\item
True. According to the part ($ii$) of the proof of Theorem 5.3.1 in the reader, we know if there exists $t_f>0$ such that $W(0, t_f)$ is singular, then we have $\text{rank}(\mathbb{C})<n$. This implies that if $\text{rank}(\mathbb{C})=n$, then $\forall t_f>0$, $W(0, t_f)$ is nonsingular. So, we know that if a LTI model is controllable, then $t_f$ can be any positive real number. For this specific problem, we choose $t_f=1$. Then, we can find a control $u(t)$ which will direct the model from $x_0$ to $0$ when $t=0$. For $t \geq 1$, set $u(t) = 0$ such that $\dot{x} = 0$, and thus the model rest on $0$.
\item
False. Let $A = \begin{bmatrix}1&2\\3&4\end{bmatrix}$, and $B = \begin{bmatrix}1&0\\0&0\end{bmatrix}$. Then, we have $\text{rank}(\mathbb{C})=2$, and model is controllable. However, for $x_f = [1,1]^T$, there doesn't exist a $u$ such that $\dot{x} = Ax_f+Bu=0$. Therefore, the model can not rest on $x_f$.
\end{enumerate}
%%%%%%%%%%%%%%%%%%%%%%%%%%%%%%%%%%%%%%%%%%%%%%%%%%%%%%%%%%%%%%%%%%

\pagebreak
\section*{Problem 3}

Define the operator $\mc{A}$ as follows:
\[
\mc{A}(u) = \int_{t_0}^{t_f} \phi(t_f,\tau) B(\tau) u(\tau) d\tau
\]
The domain of $\mc{A}$ is the set of all piecewise continuous time functions on $[t_0,t_f]$. The co-domain is $\mb{R}^n$.

\begin{enumerate}[label=\alph*)]
\item
Compute the adjoint $\mc{A}^*$.
\item
Compute the composition $V = \mc{A} \circ \mc{A}^*$. Note $V: \mb{R}^n \rightarrow \mb{R}^n$, so $V$ should be a $n \times n$ matrix. What is the relationship with $V$ and the controllability Grammian $W$? Adapt Theorem 5.2.2 from the reader to a statement about $V$.
\end{enumerate}

%%%%%%%%%%%%%%%%%%%%%%%%%%%%%%%%%%%%%%%%%%%%%%%%%%%%%%%%%%%%%%%%%%

\subsection*{Solution}
\begin{enumerate}[label=\alph*)]
\item
According to the definition of adjoint, for all $x \in \mb{R}^n$ and $u$, a piecewise continuous function on $[t_0,t_f]$, $\langle\mc{A}(u), x\rangle = \langle u, \mc{A}^*(x)\rangle$.
\begin{multline*}
  \langle\mc{A}(u), x\rangle = (\mc{A}(u))^* x = \int_{t_0}^{t_f} u^*(\tau) B^*(\tau) \phi^*(t_f,\tau) x d\tau\\
  = \langle u, \mc{A}^*(x)\rangle = \int_{t_0}^{t_f} u^*(\tau) \mc{A}^*(x)(\tau) d\tau
\end{multline*}
Because $x$ and $u$ are chosen arbitrarily, we have $\mc{A}^*(x)(t) = B^*(t) \phi^*(t_f,t) x,~t \in [t_0, t_f]$.
\item
\begin{multline*}
V(x) = \mc{A}(\mc{A}^*(x)) = \mc{A}(B^*(t) \phi^*(t_f,t) x)\\
= (\int_{t_0}^{t_f} \phi(t_f,\tau) B(\tau) B^*(t) \phi^*(t_f,t) d\tau) x
= \phi(t_f,t_0) W(t_0, t_f) \phi^*(t_f,t_0) x
\end{multline*}

Because $\phi(t_f,t_0)$ is invertible, we can say ``The LTV model is controllable at time $t_0$ if and only if there exists a finite time $t_f>t_0$ such that $\forall x \neq \upvartheta$, $V(x) \neq \upvartheta$."
\end{enumerate}
%%%%%%%%%%%%%%%%%%%%%%%%%%%%%%%%%%%%%%%%%%%%%%%%%%%%%%%%%%%%%%%%%%
%%%%%%%%%%%%%%%%%%%%%%%%%%%%%%%%%%%%%%%%%%%%%%%%%%%%%%%%%%%%%%%%%%
%%%%%%%%%%%%%%%%%%%%%%%%%%%%%%%%%%%%%%%%%%%%%%%%%%%%%%%%%%%%%%%%%%

\end{document}
