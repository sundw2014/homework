\documentclass[11pt]{report}

%%%%%%%%%%%%%%%%%%%%%%%%%%%%%%%%%%%%%%%%%%%%%%%%%%%%%%%%%%%%%%%%%%
% Set the margins of our document.
\usepackage[margin = 1 in]{geometry}

%%%%%%%%%%%%%%%%%%%%%%%%%%%%%%%%%%%%%%%%%%%%%%%%%%%%%%%%%%%%%%%%%%
% Import commands for custom header.
\usepackage{fancyhdr}
\pagestyle{fancy}

%%%%%%%%%%%%%%%%%%%%%%%%%%%%%%%%%%%%%%%%%%%%%%%%%%%%%%%%%%%%%%%%%%
% Allow ourselves to do equations!
\usepackage{amsmath,amssymb,amsthm}

%%%%%%%%%%%%%%%%%%%%%%%%%%%%%%%%%%%%%%%%%%%%%%%%%%%%%%%%%%%%%%%%%%
% Nicer formatting for enumerate commands.
\usepackage[shortlabels]{enumitem}

%%%%%%%%%%%%%%%%%%%%%%%%%%%%%%%%%%%%%%%%%%%%%%%%%%%%%%%%%%%%%%%%%%
% Colored text and include images.
\usepackage{color}
\usepackage{graphicx}

%%%%%%%%%%%%%%%%%%%%%%%%%%%%%%%%%%%%%%%%%%%%%%%%%%%%%%%%%%%%%%%%%%
% Some custom macros to make life easier.
\newcommand{\mc}{\mathcal}
\newcommand{\mb}{\mathcal}

\newcommand{\T}{\intercal}

%%%%%%%%%%%%%%%%%%%%%%%%%%%%%%%%%%%%%%%%%%%%%%%%%%%%%%%%%%%%%%%%%%
%%%%%%%%%%%%%%%%%%%%%%%%%%%%%%%%%%%%%%%%%%%%%%%%%%%%%%%%%%%%%%%%%%
%%%%%%%%%%%%%%%%%%%%%%%%%%%%%%%%%%%%%%%%%%%%%%%%%%%%%%%%%%%%%%%%%%

\lhead{ECE 515 - Fall 2019 at University of Illinois at Urbana-Champaign}
\rhead{\textcolor{red}{HW6 - Template}}
\lfoot{Submitted by: \textcolor{red}{\textit{daweis2}}}
\rfoot{Due: Thursday, October 31 at 2pm}

%%%%%%%%%%%%%%%%%%%%%%%%%%%%%%%%%%%%%%%%%%%%%%%%%%%%%%%%%%%%%%%%%%
%%%%%%%%%%%%%%%%%%%%%%%%%%%%%%%%%%%%%%%%%%%%%%%%%%%%%%%%%%%%%%%%%%
%%%%%%%%%%%%%%%%%%%%%%%%%%%%%%%%%%%%%%%%%%%%%%%%%%%%%%%%%%%%%%%%%%

\begin{document}

%%%%%%%%%%%%%%%%%%%%%%%%%%%%%%%%%%%%%%%%%%%%%%%%%%%%%%%%%%%%%%%%%%
%%%%%%%%%%%%%%%%%%%%%%%%%%%%%%%%%%%%%%%%%%%%%%%%%%%%%%%%%%%%%%%%%%
%%%%%%%%%%%%%%%%%%%%%%%%%%%%%%%%%%%%%%%%%%%%%%%%%%%%%%%%%%%%%%%%%%

%%%%%%%%%%%%%%%%%%%%%%%%%%%%%%%%%%%%%%%%%%%%%%%%%%%%%%%%%%%%%%%%%%
%%%%%%%%%%%%%%%%%%%%%%%%%%%%%%%%%%%%%%%%%%%%%%%%%%%%%%%%%%%%%%%%%%
%%%%%%%%%%%%%%%%%%%%%%%%%%%%%%%%%%%%%%%%%%%%%%%%%%%%%%%%%%%%%%%%%%
\section*{Problem 1}

Consider the linear, time-varying system:
\[
\dot x(t) = A(t) x(t) + B(t)u(t)
\]
\[
y(t) = C(t) u(t)
\]
Recall the definition of the observability Grammian:
\[
H(t_1,t_0) = \int_{t_0}^{t_1} \phi^\T(\tau, t_0) C^\T(\tau) C(\tau) \phi(\tau, t_0) d\tau
\]
Consider the function from $\mb{R}$ to $\mb{R}^{n \times n}$:
\[
X : t_0 \mapsto H(t_1,t_0)
\]
(Two other ways to write this are: $X(t_0) = H(t_1,t_0)$, or $X(\cdot) = H(t_1,\cdot)$.)

Show that the function $X$ satisfies the linear matrix differential equation:
\[
\dot X(t) = -A^\T(t) X(t) - X(t) A(t) - C^\T(t) C(t) \qquad \qquad X(t_1) = 0_{n \times n}
\]
Here, the initial condition $0_{n \times n}$ is the zero matrix in $\mb{R}^{n \times n}$.

%%%%%%%%%%%%%%%%%%%%%%%%%%%%%%%%%%%%%%%%%%%%%%%%%%%%%%%%%%%%%%%%%%

\subsection*{Solution}
Denoting $$X(t) = \int_{t}^{t_1} \phi^T(\tau, t)C^T(\tau)C(\tau)\phi(\tau, t)d\tau = \int_{t}^{t_1} f(t, \tau) d\tau$$
Using the Leibniz rule:
\begin{multline}
  \dot{X}(t) = -f(t,t) + \int_{t}^{t_1}\frac{\partial f(t,\tau)}{\partial t} d\tau\\
  = -I C^\T(t) C(t) I + \int_{t}^{t_1}\left[-A(t)^\T\phi^\T(\tau, t)C^\T(\tau)C(\tau)\phi(\tau, t) + \phi^\T(\tau, t)C^\T(\tau)C(\tau)\phi(\tau, t)(-A(t))\right] d\tau\\
  = -A^\T(t) X(t) - X(t) A(t) - C^\T(t) C(t)
\end{multline}
 and we have
$$X(t_1) = \int_{t_1}^{t_1} f(t, \tau) d\tau = 0_{n \times n}$$

%%%%%%%%%%%%%%%%%%%%%%%%%%%%%%%%%%%%%%%%%%%%%%%%%%%%%%%%%%%%%%%%%%

\pagebreak
\section*{Problem 2}

Consider a linear time-varying system with dynamics:
\[
\dot x(t) = A(t) x(t) + B(t) u(t)
\]
\[
y(t) = C(t) x(t)
\]
Let's call this system $R$.

As we've covered before, the dual system is given by the dynamics:
\[
\dot{\tilde{x}}(t) = -A^\T(t)\tilde x(t) - C^\T(t) \tilde u(t)
\]
\[
\tilde y(t) = B^\T(t) \tilde x(t)
\]
Let's call this dual system $\tilde R$.

Consider any state $x_0$ that is controllable to zero on $[t_0,t_1]$ for $R$, and any state $\tilde x_0$ that is unobservable on $[t_0,t_1]$ for $\tilde R$. Show that $x_0$ and $\tilde x_0$ are orthogonal, i.e. $\langle x_0, \tilde x_0 \rangle = 0$.

%%%%%%%%%%%%%%%%%%%%%%%%%%%%%%%%%%%%%%%%%%%%%%%%%%%%%%%%%%%%%%%%%%

\subsection*{Solution}
Because $x_0$ is controllable to zero, there exists a $u(t)$ such that $x_0 = \int_{t_0}^{t_1} \phi(t_0, \tau) B(\tau) u(\tau) d\tau$. Because $\tilde x_0$ is unobservable, $B^\T(t) \phi^\T(t_0, t) \tilde x_0 = 0,~\forall t \in [t_0, t_1]$. Therefore, $\langle x_0, \tilde x_0 \rangle = x_0^\T \tilde x_0 = \int_{t_0}^{t_1} u^\T(\tau) B^\T(\tau) \phi^\T(t_0, \tau) \tilde x_0 d\tau = 0$.
%%%%%%%%%%%%%%%%%%%%%%%%%%%%%%%%%%%%%%%%%%%%%%%%%%%%%%%%%%%%%%%%%%

\pagebreak
\section*{Problem 3}



Consider:
\[
A =
\begin{bmatrix}
-1 & 0 \\
0 & -1
\end{bmatrix}
\qquad
B =
\begin{bmatrix}
-1 \\
1
\end{bmatrix}
\qquad
C =
\begin{bmatrix}
1 & 0 \\
0 & 1
\end{bmatrix}
\]
\begin{itemize}
\item Is the system controllable? If not, put it in Kalman controllability canonical form.
\item Is the system observable? If not, put it in Kalman observability canonical form.
\end{itemize}

%%%%%%%%%%%%%%%%%%%%%%%%%%%%%%%%%%%%%%%%%%%%%%%%%%%%%%%%%%%%%%%%%%

\subsection*{Solution}
\begin{itemize}
\item Controallability matrix $\mathcal{C} = \begin{bmatrix} -1 & 1 \\1 & -1 \end{bmatrix}$. Since $\text{rank}(\mathcal{C}) = 1$, it is uncontrollable. Let $P^{-1} = \begin{bmatrix} -1 & 1 \\1 & 0 \end{bmatrix}$. Then, $\bar{A} = P A P^{-1} = \begin{bmatrix} -1 & 0 \\ 0 & -1 \end{bmatrix} \quad \bar{B} = PB = \begin{bmatrix} 1 \\ 0 \end{bmatrix} \quad \bar{C} = CP^{-1} = \begin{bmatrix} -1 & 1 \\1 & 0 \end{bmatrix}$.
\item Observablity matrix $\mathcal{O} = \begin{bmatrix} 1 & 0 \\0 & 1\\-1 & 0\\0 & -1 \end{bmatrix}$. Since $\text{rank}(\mathcal{O}) = 2$, it is observable.
\end{itemize}
%%%%%%%%%%%%%%%%%%%%%%%%%%%%%%%%%%%%%%%%%%%%%%%%%%%%%%%%%%%%%%%%%%

\pagebreak
\section*{Problem 4}

Take the transfer function:
\[
H(s) = \frac{s+3}{(s+1)(s+2)}
\]
\begin{itemize}
\item Put this system into controllable canonical form.
\item Using static linear state feedback ($u = -Kx$), find a $K$ that places the poles at $-5 \pm 2j$.
\end{itemize}

%%%%%%%%%%%%%%%%%%%%%%%%%%%%%%%%%%%%%%%%%%%%%%%%%%%%%%%%%%%%%%%%%%

\subsection*{Solution}
\begin{itemize}
\item $H(s) = \frac{s+3}{s^2+3s+2}$. CCF: $A = \begin{bmatrix} 0 & 1 \\-2 & -3 \end{bmatrix} \quad B = \begin{bmatrix} 0 \\ 1 \end{bmatrix} \quad C = \begin{bmatrix} 3 \\ 1 \end{bmatrix}$.
\item Let $K = [k_1 ~ k_2]$, then $A - B K = \begin{bmatrix} 0 & 1 \\-(2 + k_1) & -(3 + k_2) \end{bmatrix}$ and the closed-loop characteristic polynomial is $\Delta(s) = s^2 + (3 + k_2)s + (2 + k_1)$. Then substitute the roots, and we have $K = \begin{bmatrix}27 & 7\end{bmatrix}$.
\end{itemize}

%%%%%%%%%%%%%%%%%%%%%%%%%%%%%%%%%%%%%%%%%%%%%%%%%%%%%%%%%%%%%%%%%%

\pagebreak
\section*{Problem 5}

Consider the following dynamical system, inspired by a linearization of the pendubot from Chapter 1 in the reader.
\[
\dot x =
\begin{bmatrix}
0 & 1 & 0 & 0 \\
5 & 0 & -2 & 0 \\
0 & 0 & 0 & 1 \\
-5 & 6 & 0 & 0
\end{bmatrix}
x +
\begin{bmatrix}
0 \\
2 \\
0 \\
-3
\end{bmatrix}
u
\]
\[
y =
\begin{bmatrix}
1 & 0 & 0 & 0
\end{bmatrix}
x
\]

Design a reduced-order Luenberger observer for this system. You may freely use MATLAB or any other computer assistance to do so (and are encouraged to do so!), but still show your work.

%%%%%%%%%%%%%%%%%%%%%%%%%%%%%%%%%%%%%%%%%%%%%%%%%%%%%%%%%%%%%%%%%%

\subsection*{Solution}
Using the notations in the reader, we have $A_{11} = 0$, $A_{21} = \begin{bmatrix}5\\ 0\\ -5\end{bmatrix}$, $A_{12} = \begin{bmatrix}1 & 0 & 0\end{bmatrix}$, $A_{22} = \begin{bmatrix}0 & -2 & 0\\0 & 0 & 1\\6 & 0 & 0\end{bmatrix}$, $B_1 = 0$, $B_2 = \begin{bmatrix} 2\\ 0\\ -3\end{bmatrix}$ and $y=x_1$, $x = \begin{bmatrix}x_1\\ x_2\end{bmatrix}$.

Then, the matrix for the observer is $A_{22} - L A_{12} = \begin{bmatrix}-\ell_1 & -2 & 0\\ -\ell_2 & 0 & 1\\ 6 - \ell_3 & 0 & 0\end{bmatrix}$. The characteristic polynomial is $\Delta (s) = s^3 + \ell_1 s^2 - 2\ell_2 s - 2\ell_3 + 12$. Putting all the poles at $-1$, we have $L = \begin{bmatrix}3 & -\frac{3}{2} & \frac{11}{2}\end{bmatrix}^\T$.


The observer is $\dot{\hat{x}}_2 = A_{22} \hat{x}_2 + A_{21} y + B_2 u + L(\dot{y} - A_{12}\hat{x}_2)$. Then, $x$ can be recovered from $y$ and $x_2$, i.e. $\hat{x} = \begin{bmatrix}y\\ \hat{x}_2\end{bmatrix}$.
%%%%%%%%%%%%%%%%%%%%%%%%%%%%%%%%%%%%%%%%%%%%%%%%%%%%%%%%%%%%%%%%%%
%%%%%%%%%%%%%%%%%%%%%%%%%%%%%%%%%%%%%%%%%%%%%%%%%%%%%%%%%%%%%%%%%%
%%%%%%%%%%%%%%%%%%%%%%%%%%%%%%%%%%%%%%%%%%%%%%%%%%%%%%%%%%%%%%%%%%

\end{document}
