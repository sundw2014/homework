\documentclass[11pt]{report}

%%%%%%%%%%%%%%%%%%%%%%%%%%%%%%%%%%%%%%%%%%%%%%%%%%%%%%%%%%%%%%%%%%
% Set the margins of our document.
\usepackage[margin = 1 in]{geometry}

%%%%%%%%%%%%%%%%%%%%%%%%%%%%%%%%%%%%%%%%%%%%%%%%%%%%%%%%%%%%%%%%%%
% Import commands for custom header.
\usepackage{fancyhdr}
\pagestyle{fancy}

%%%%%%%%%%%%%%%%%%%%%%%%%%%%%%%%%%%%%%%%%%%%%%%%%%%%%%%%%%%%%%%%%%
% Allow ourselves to do equations!
\usepackage{amsmath,amssymb,amsthm}

%%%%%%%%%%%%%%%%%%%%%%%%%%%%%%%%%%%%%%%%%%%%%%%%%%%%%%%%%%%%%%%%%%
% Nicer formatting for enumerate commands.
\usepackage[shortlabels]{enumitem}

%%%%%%%%%%%%%%%%%%%%%%%%%%%%%%%%%%%%%%%%%%%%%%%%%%%%%%%%%%%%%%%%%%
% Colored text and include images.
\usepackage{color}
\usepackage{graphicx}

%%%%%%%%%%%%%%%%%%%%%%%%%%%%%%%%%%%%%%%%%%%%%%%%%%%%%%%%%%%%%%%%%%
% Some custom macros to make life easier.
\newcommand{\mc}{\mathcal}
\newcommand{\mb}{\mathbb}

\newcommand{\T}{\intercal}

\usepackage{tikz}
\usetikzlibrary{positioning,arrows}

%%%%%%%%%%%%%%%%%%%%%%%%%%%%%%%%%%%%%%%%%%%%%%%%%%%%%%%%%%%%%%%%%%
%%%%%%%%%%%%%%%%%%%%%%%%%%%%%%%%%%%%%%%%%%%%%%%%%%%%%%%%%%%%%%%%%%
%%%%%%%%%%%%%%%%%%%%%%%%%%%%%%%%%%%%%%%%%%%%%%%%%%%%%%%%%%%%%%%%%%

\lhead{ECE 515 - Fall 2019 at University of Illinois at Urbana-Champaign}
\rhead{\textcolor{red}{HW7 - Template}}
\lfoot{Submitted by: \textcolor{red}{\textit{daweis2}}}
\rfoot{Due: Thursday, November 7 at 2pm}

%%%%%%%%%%%%%%%%%%%%%%%%%%%%%%%%%%%%%%%%%%%%%%%%%%%%%%%%%%%%%%%%%%
%%%%%%%%%%%%%%%%%%%%%%%%%%%%%%%%%%%%%%%%%%%%%%%%%%%%%%%%%%%%%%%%%%
%%%%%%%%%%%%%%%%%%%%%%%%%%%%%%%%%%%%%%%%%%%%%%%%%%%%%%%%%%%%%%%%%%

\begin{document}

\tikzset{%
    block/.style={draw, fill=white, rectangle,
            minimum height=2em, minimum width=3em},
    input/.style={inner sep=0pt},
    output/.style={inner sep=0pt},
    sum/.style = {draw, fill=white, circle, minimum size=2mm, node distance=1.5cm, inner sep=0pt},
    pinstyle/.style = {pin edge={to-,thin,black}}
}

%%%%%%%%%%%%%%%%%%%%%%%%%%%%%%%%%%%%%%%%%%%%%%%%%%%%%%%%%%%%%%%%%%
%%%%%%%%%%%%%%%%%%%%%%%%%%%%%%%%%%%%%%%%%%%%%%%%%%%%%%%%%%%%%%%%%%
%%%%%%%%%%%%%%%%%%%%%%%%%%%%%%%%%%%%%%%%%%%%%%%%%%%%%%%%%%%%%%%%%%

%\pagebreak
\section*{Problem 1}

Let
\[
\dot x =
\begin{bmatrix}
0 & 1 \\
5 & 4
\end{bmatrix}
x +
\begin{bmatrix}
0 \\ 1
\end{bmatrix}
u
\]
\[
y = \begin{bmatrix} 1 & 0 \end{bmatrix} x = x_1
\]
\begin{itemize}
\item
Draw a block diagram representing this system.
\item
Design a reduced-order Luenberger observer, and draw the block diagram for the system and the observer.
\end{itemize}

%%%%%%%%%%%%%%%%%%%%%%%%%%%%%%%%%%%%%%%%%%%%%%%%%%%%%%%%%%%%%%%%%%

\subsection*{Solution}
\begin{itemize}
\item
\begin{tikzpicture}[auto, node distance=2cm, on grid, >=latex']

\node [input] (input) {};
\node [sum, right = of input] (sum) {$\sum$};
\node [block, right = of sum] (int1) {$\int$};
\node [block, right = 3cm of int1] (int2) {$\int$};
\node [block, below = 2cm of int1] (amp1) {$4$};
\node [block, below = 3cm of int2] (amp2) {$5$};
\node [output, right = of int2] (output) {};

\draw [draw,->] (input) node[above right] {$u$} -- (sum);
\draw [->] (sum) -- node[name=x2dot, pos=.3] {$\dot{x_2}$} (int1);
\draw [->] (int1) -- node[name=x2, pos=.3] {$x_2$} (int2);
\draw [->] (int2) -- node[name=y, pos=.7] {$y$} (output);
\draw [->] (x2.east|-int2) |- (amp1);
\draw [->] (amp1) -| (sum);
\draw [->] (y.west|-int2) |- (amp2);
\draw [->] (amp2) -| (sum);
%
\end{tikzpicture}

\item
Using the notations in the reader, we have $A_{11} = 0$, $A_{12} = 1$, $A_{21} = 5$, $A_{22} = 4$, $B_1 = 0$, $B_2 = 1$ and $y=x_1$, $x = \begin{bmatrix}x_1\\ x_2\end{bmatrix}$.

Considering the gain of the observer feedback $L$, we must have $A_{22} - L A_{12} = 4 - L < 0$. Choose $L = 5$. Then, the observer equation is
$$ \dot{\hat{x}}_2 = 4\hat{x}_2 + 5 y + u + L(\dot{y} - \hat{x}_2) = -\hat{x}_2 + 5y + u + 5\dot{y}$$

\begin{tikzpicture}[auto, node distance=2cm, on grid, >=latex']

\node [input] (input1) {};
\node [sum, right = of input1] (sum1) {$\sum$};
\node [block, right = of sum1] (int1) {$\int$};
\node [sum, right = of int1] (sum2) {$\sum$};
\node [block, right = of sum] (int1) {$\int$};
\node [block, below = of int1] (amp1) {$-1$};
\node [output, right = of sum2] (output) {};
\node [input, above = of input1] (input2) {};
\node [block, right = of input2] (amp2) {$5$};

\draw [draw,->] (input1) node[above right] {$u$} -- (sum1);
\draw [->] (sum1) -- (int1);
\draw [->] (int1) -- (sum2);
\draw [->] (sum2) -- node[name=hatx_2, pos=.7] {$\hat{x}_2$} (output);
\draw [->] (hatx_2.west|-sum2) |- (amp1);
\draw [->] (amp1) -| (sum1);
\draw [->] (input2) node[above right] {$y$} -- (amp2);
\draw [->] (amp2) -- (sum1);
\draw [->] (amp2) -| (sum2);
%
\end{tikzpicture}

\end{itemize}
%%%%%%%%%%%%%%%%%%%%%%%%%%%%%%%%%%%%%%%%%%%%%%%%%%%%%%%%%%%%%%%%%%
%%%%%%%%%%%%%%%%%%%%%%%%%%%%%%%%%%%%%%%%%%%%%%%%%%%%%%%%%%%%%%%%%%
%%%%%%%%%%%%%%%%%%%%%%%%%%%%%%%%%%%%%%%%%%%%%%%%%%%%%%%%%%%%%%%%%%

\pagebreak
\section*{Problem 2}

Take any matrix $A \in \mb{R}^{m \times n}$. Prove the following statements:
\begin{itemize}
\item $\mc{R}(A)^\bot = \mc{N}(A^\T)$
\item $\mc{N}(A)^\bot = \mc{R}(A^\T)$
\end{itemize}

%%%%%%%%%%%%%%%%%%%%%%%%%%%%%%%%%%%%%%%%%%%%%%%%%%%%%%%%%%%%%%%%%%

\subsection*{Solution}
\begin{itemize}
\item
($\Rightarrow$) If $\alpha$ is an element of $\mc{R}(A)^\bot$, then for any $x \in \mb{R}^n,~Ax \in \mc{R}(A)$, we have $\alpha^\T A x = 0$. Because this equation holds for all $x$ in $\mb{R}^n$, we must have $\alpha^\T A = 0$ and $\alpha \in \mc{N}(A^\T)$.

($\Leftarrow$) If $\alpha$ is an element of $\mc{N}(A^\T)$, then $A^\T \alpha = 0$. For all $x \in \mb{R}^n,~Ax \in \mc{R}(A)$, we have $(A x)^\T \alpha  = x^\T A^\T \alpha = 0$ and $\alpha \in \mc{R}(A)^\bot$.

\item
$\mc{N}(A)^\bot = (\mc{R}(A^\T)^\bot)^\bot = \mc{R}(A^\T)$.
\end{itemize}
%%%%%%%%%%%%%%%%%%%%%%%%%%%%%%%%%%%%%%%%%%%%%%%%%%%%%%%%%%%%%%%%%%
%%%%%%%%%%%%%%%%%%%%%%%%%%%%%%%%%%%%%%%%%%%%%%%%%%%%%%%%%%%%%%%%%%
%%%%%%%%%%%%%%%%%%%%%%%%%%%%%%%%%%%%%%%%%%%%%%%%%%%%%%%%%%%%%%%%%%

\pagebreak
\section*{Problem 3}

Consider the following dynamics:
\[
A =
\begin{bmatrix}
1 & 0 & 0 & 0 \\
0 & 2 & 0 & 0 \\
0 & 0 & 3 & 0 \\
0 & 0 & 0 & 4
\end{bmatrix}
\qquad
B =
\begin{bmatrix}
1 \\ 1 \\ 0 \\ 0
\end{bmatrix}
\qquad
C =
\begin{bmatrix}
1 & 0 & 1 & 0
\end{bmatrix}
\]
In this problem, you will be asked to calculate two Kalman decompositions for this system.
\begin{itemize}
\item Calculate $\Sigma_{c \bar o} = \mc{R}(\mc{C}) \cap \mc{N}(\mc{O})$
\item Calculate a $\Sigma_{c o}$ such that $\Sigma_{c \bar o} \oplus \Sigma_{c o} = \mc{R}(\mc{C})$
\item Calculate a $\Sigma_{\bar c \bar o}$ such that $\Sigma_{c \bar o} \oplus \Sigma_{\bar c \bar o} = \mc{N}(\mc{O})$
\item Calculate a $\Sigma_{\bar c o}$ such that $\Sigma_{c \bar o} \oplus \Sigma_{c o} \oplus \Sigma_{\bar c \bar o} \oplus \Sigma_{\bar c o} = \mb{R}^n$
\item As we discussed in class, this need not be unique; calculate a different $\Sigma_{c o}$, $\Sigma_{\bar c \bar o}$, and $\Sigma_{\bar c o}$ for the same system
\end{itemize}

%%%%%%%%%%%%%%%%%%%%%%%%%%%%%%%%%%%%%%%%%%%%%%%%%%%%%%%%%%%%%%%%%%

\subsection*{Solution}
We have $\mc{C} = \begin{bmatrix}1 & 1 & 1 & 1\\ 1 & 2 & 4 & 8\\ 0 & 0 & 0 & 0\\ 0 & 0 & 0 & 0\end{bmatrix}$ and $\mc{O} = \begin{bmatrix}1 & 0 & 1 & 0\\ 1 & 0 & 3 & 0\\ 1 & 0 & 9 & 0\\ 1 & 0 & 27 & 0\end{bmatrix}$. Let $e_1,~e_2,~e_3,~e_4$ be the standard basis of $R^4$. Then, we have $\mc{R}(\mc{C}) = <e_1, e_2>,~\mc{N}(\mc{O}) = <e_2, e_4>$, where $<e_1, e_2>$ refers to the subspace spanned by $e_1$ and $e_2$.
\begin{itemize}
\item $\Sigma_{c \bar o} = \mc{R}(\mc{C}) \cap \mc{N}(\mc{O}) = <e_2>$
\item $\Sigma_{c o} = <e_1>$
\item $\Sigma_{\bar c \bar o} = <e_4>$
\item $\Sigma_{\bar c o} = <e_3>$
\item $\Sigma_{c \bar o} = <e_2>,~\Sigma_{c o} = <e_1+e_2>,~\Sigma_{\bar c \bar o} = <e_2+e_4>,~\Sigma_{\bar c o} = <e_1+e_2+e_3+e_4>$
\end{itemize}

%%%%%%%%%%%%%%%%%%%%%%%%%%%%%%%%%%%%%%%%%%%%%%%%%%%%%%%%%%%%%%%%%%
%%%%%%%%%%%%%%%%%%%%%%%%%%%%%%%%%%%%%%%%%%%%%%%%%%%%%%%%%%%%%%%%%%
%%%%%%%%%%%%%%%%%%%%%%%%%%%%%%%%%%%%%%%%%%%%%%%%%%%%%%%%%%%%%%%%%%

\end{document}
