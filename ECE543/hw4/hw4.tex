\documentclass[11pt]{report}
% this template is originally from Roy Dong's ECE 515.
% Editted by Dawei Sun
%%%%%%%%%%%%%%%%%%%%%%%%%%%%%%%%%%%%%%%%%%%%%%%%%%%%%%%%%%%%%%%%%%
% Set the margins of our document.
\usepackage[margin = 1 in]{geometry}

%%%%%%%%%%%%%%%%%%%%%%%%%%%%%%%%%%%%%%%%%%%%%%%%%%%%%%%%%%%%%%%%%%
% Import commands for custom header.
\usepackage{fancyhdr}
\pagestyle{fancy}

%%%%%%%%%%%%%%%%%%%%%%%%%%%%%%%%%%%%%%%%%%%%%%%%%%%%%%%%%%%%%%%%%%
% Allow ourselves to do equations!
\usepackage{amsmath,amssymb,amsthm}
\usepackage{bm}
\usepackage{upgreek}

%%%%%%%%%%%%%%%%%%%%%%%%%%%%%%%%%%%%%%%%%%%%%%%%%%%%%%%%%%%%%%%%%%
% Nicer formatting for enumerate commands.
\usepackage[shortlabels]{enumitem}

%%%%%%%%%%%%%%%%%%%%%%%%%%%%%%%%%%%%%%%%%%%%%%%%%%%%%%%%%%%%%%%%%%
% Colored text and include images.
\usepackage{color}
\usepackage{graphicx}
\usepackage{mathtools}
\usepackage{bbm}

\usepackage{hyperref}
\hypersetup{
    colorlinks=true,
    linkcolor=blue,
    filecolor=magenta,
    urlcolor=cyan,
}

\urlstyle{same}

%%%%%%%%%%%%%%%%%%%%%%%%%%%%%%%%%%%%%%%%%%%%%%%%%%%%%%%%%%%%%%%%%%
% Some custom macros to make life easier.
\DeclarePairedDelimiter\ceil{\lceil}{\rceil}
\DeclarePairedDelimiter\floor{\lfloor}{\rfloor}
\newcommand{\mc}{\mathcal}
\newcommand{\mb}{\mathbb}
\newcommand{\vect}[1]{\boldsymbol{\mathbf{#1}}}
\newcommand{\T}{\intercal}
\newcommand{\E}[1]{\mathbb{E}\left[#1\right]}
\newcommand{\condi}[2]{#1 \ | \ #2}
%%%%%%%%%%%%%%%%%%%%%%%%%%%%%%%%%%%%%%%%%%%%%%%%%%%%%%%%%%%%%%%%%%
%%%%%%%%%%%%%%%%%%%%%%%%%%%%%%%%%%%%%%%%%%%%%%%%%%%%%%%%%%%%%%%%%%
%%%%%%%%%%%%%%%%%%%%%%%%%%%%%%%%%%%%%%%%%%%%%%%%%%%%%%%%%%%%%%%%%%

\lhead{ECE 543 - Spring 2020 at University of Illinois at Urbana-Champaign}
\rhead{\textcolor{red}{HOMEWORK4}}
\lfoot{Submitted by: Dawei Sun (\textcolor{red}{\textit{daweis2}})}

%%%%%%%%%%%%%%%%%%%%%%%%%%%%%%%%%%%%%%%%%%%%%%%%%%%%%%%%%%%%%%%%%%
%%%%%%%%%%%%%%%%%%%%%%%%%%%%%%%%%%%%%%%%%%%%%%%%%%%%%%%%%%%%%%%%%%
%%%%%%%%%%%%%%%%%%%%%%%%%%%%%%%%%%%%%%%%%%%%%%%%%%%%%%%%%%%%%%%%%%

\begin{document}

%%%%%%%%%%%%%%%%%%%%%%%%%%%%%%%%%%%%%%%%%%%%%%%%%%%%%%%%%%%%%%%%%%
%%%%%%%%%%%%%%%%%%%%%%%%%%%%%%%%%%%%%%%%%%%%%%%%%%%%%%%%%%%%%%%%%%
%%%%%%%%%%%%%%%%%%%%%%%%%%%%%%%%%%%%%%%%%%%%%%%%%%%%%%%%%%%%%%%%%%

%\pagebreak
\section*{Problem 1}
\subsection*{Solution}
\paragraph{Symmetric} $K(x,y) = \langle K(x,\cdot), K(y,\cdot) \rangle = \overline{\langle K(y,\cdot), K(x,\cdot) \rangle} = \overline{K(y,x)} = K(y,x)$.
\paragraph{Positive semi-definite} For any $n \in \mathbb{N}^+$ and $x_1,\ x_2,\ \cdots,\ x_n \in X$, construct a matrix $M$ such that $M_{i,j} = K(x_i, x_j)$. For all $v \in \mathbb{R}^n$, we have $v^\T M v = \langle \sum_{i=1}^{n} v_i K(x_i, \cdot), \sum_{i=1}^{n} v_i K(x_i, \cdot) \rangle \geq 0$. Therefore, $M$ is positive semi-definite.

\section*{Problem 2}
\subsection*{Solution}
\begin{enumerate}[(a)]
  \item For all $x,\ y \in X$, construct a matrix $M = \begin{bmatrix}K(x,x) & K(x,y)\\ K(x,y) & K(y,y)\end{bmatrix}$. Then, $M$ is positive semi-definite. Thus, $\det (M) = K(x,x)K(y,y) - K(x,y)^2 \geq 0$.
  \item (1) $\langle f,g \rangle = \sum_{i=1}^{n}\sum_{j=1}^{m} \alpha_i \beta_j K(x_i, x_j) = \sum_{i=1}^{n}\sum_{j=1}^{m} \alpha_i \beta_j K(x_j, x_i) = \langle g,f \rangle = \overline{\langle g,f \rangle}$; (2) For all $a,b \in \mathbb{R}$ and $f = \sum_{i=1}^{n} \alpha_i K(x_i, \cdot)$, $g = \sum_{i=1}^{m} \beta_i K(x_i, \cdot)$, $h = \sum_{i=1}^{k} \gamma_i K(x_i, \cdot)$, without loss of generality, suppose that $m \leq n$, and set $\beta_{m+1},\cdots,\beta_{n} = 0$. We have $\langle af+bg, h \rangle = \sum_{i=1}^{n}\sum_{j=1}^{k} (a\alpha_i + b \beta_i)\gamma_j K(x_i, x_j) = a\sum_{i=1}^{n}\sum_{j=1}^{k} \alpha_i \gamma_j K(x_i, x_j) + b\sum_{i=1}^{m}\sum_{j=1}^{k} \beta_i \gamma_j K(x_i, x_j) = a \langle f,h \rangle + b \langle g,h \rangle$; (3) For all $f = \sum_{i=1}^{n} \alpha_i K(x_i, \cdot)$, construct matrix $M$ such that $M_{i,j} = K(x_i, x_j)$. Then, $\langle f,f \rangle = \alpha^\T M \alpha \geq 0$. Next, we prove $\langle f,f \rangle = 0 \iff f \equiv 0$. It is obvious that if $f \equiv 0$, $\langle f,f \rangle = 0$. If $\langle f,f \rangle = 0$, then for all $x \in X$, $|f(x)|^2 = \langle f, K(x,\cdot) \rangle ^ 2 \leq \langle f,f \rangle \langle K(x,\cdot), K(x,\cdot) \rangle = 0$, and thus $f(x) = 0$. Note that here Cauchy-Schwarz inequality holds since it doesn't rely on the property we are currently proving. (I borrowed this result from page 37 in
  \url{https://www.ism.ac.jp/~fukumizu/H20_kernel/Kernel_2_elements.pdf})
  % , and $\langle f,f \rangle = 0$ only if ???
\end{enumerate}

\section*{Problem 3}
\subsection*{Solution}
Let $S$ be the subspace spanned by $K(x_1,\cdot), \cdots, K(x_n, \cdot)$. Then, for all $f$ we can decomose it into $f = f_s + v$ such that $f_s \in S$ and $v \bot f_s$. We have $||f||^2 = ||f_s||^2 + ||v||^2 \geq ||f_s||^2$ and $g(||f_s||) \leq g(||f||)$. We also have, for all $x_i$, $f(x_i) = \langle f,K(x_i,\cdot) \rangle = \langle f_s,K(x_i,\cdot) \rangle + \langle v,K(x_i,\cdot) \rangle = \langle f_s,K(x_i,\cdot) \rangle = f_s(x_i)$. Thus, for a minimizer $f^*$, we must have $||v|| = 0,\ v = \upvartheta$, and $f^*$ can be linearly represented by $K(x_1,\cdot), \cdots, K(x_n, \cdot)$.
\end{document}
