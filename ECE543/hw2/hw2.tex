\documentclass[11pt]{report}
% this template is originally from Roy Dong's ECE 515.
% Editted by Dawei Sun
%%%%%%%%%%%%%%%%%%%%%%%%%%%%%%%%%%%%%%%%%%%%%%%%%%%%%%%%%%%%%%%%%%
% Set the margins of our document.
\usepackage[margin = 1 in]{geometry}

%%%%%%%%%%%%%%%%%%%%%%%%%%%%%%%%%%%%%%%%%%%%%%%%%%%%%%%%%%%%%%%%%%
% Import commands for custom header.
\usepackage{fancyhdr}
\pagestyle{fancy}

%%%%%%%%%%%%%%%%%%%%%%%%%%%%%%%%%%%%%%%%%%%%%%%%%%%%%%%%%%%%%%%%%%
% Allow ourselves to do equations!
\usepackage{amsmath,amssymb,amsthm}
\usepackage{bm}
\usepackage{upgreek}

%%%%%%%%%%%%%%%%%%%%%%%%%%%%%%%%%%%%%%%%%%%%%%%%%%%%%%%%%%%%%%%%%%
% Nicer formatting for enumerate commands.
\usepackage[shortlabels]{enumitem}

%%%%%%%%%%%%%%%%%%%%%%%%%%%%%%%%%%%%%%%%%%%%%%%%%%%%%%%%%%%%%%%%%%
% Colored text and include images.
\usepackage{color}
\usepackage{graphicx}
\usepackage{mathtools}
\usepackage{bbm}

\usepackage{hyperref}
\hypersetup{
    colorlinks=true,
    linkcolor=blue,
    filecolor=magenta,
    urlcolor=cyan,
}

\urlstyle{same}

%%%%%%%%%%%%%%%%%%%%%%%%%%%%%%%%%%%%%%%%%%%%%%%%%%%%%%%%%%%%%%%%%%
% Some custom macros to make life easier.
\DeclarePairedDelimiter\ceil{\lceil}{\rceil}
\DeclarePairedDelimiter\floor{\lfloor}{\rfloor}
\newcommand{\mc}{\mathcal}
\newcommand{\mb}{\mathbb}
\newcommand{\vect}[1]{\boldsymbol{\mathbf{#1}}}
\newcommand{\T}{\intercal}
\newcommand{\E}[1]{\mathbb{E}\left[#1\right]}
\newcommand{\condi}[2]{#1 \ | \ #2}
%%%%%%%%%%%%%%%%%%%%%%%%%%%%%%%%%%%%%%%%%%%%%%%%%%%%%%%%%%%%%%%%%%
%%%%%%%%%%%%%%%%%%%%%%%%%%%%%%%%%%%%%%%%%%%%%%%%%%%%%%%%%%%%%%%%%%
%%%%%%%%%%%%%%%%%%%%%%%%%%%%%%%%%%%%%%%%%%%%%%%%%%%%%%%%%%%%%%%%%%

\lhead{ECE 543 - Spring 2020 at University of Illinois at Urbana-Champaign}
\rhead{\textcolor{red}{HOMEWORK2}}
\lfoot{Submitted by: Dawei Sun (\textcolor{red}{\textit{daweis2}})}

%%%%%%%%%%%%%%%%%%%%%%%%%%%%%%%%%%%%%%%%%%%%%%%%%%%%%%%%%%%%%%%%%%
%%%%%%%%%%%%%%%%%%%%%%%%%%%%%%%%%%%%%%%%%%%%%%%%%%%%%%%%%%%%%%%%%%
%%%%%%%%%%%%%%%%%%%%%%%%%%%%%%%%%%%%%%%%%%%%%%%%%%%%%%%%%%%%%%%%%%

\begin{document}

%%%%%%%%%%%%%%%%%%%%%%%%%%%%%%%%%%%%%%%%%%%%%%%%%%%%%%%%%%%%%%%%%%
%%%%%%%%%%%%%%%%%%%%%%%%%%%%%%%%%%%%%%%%%%%%%%%%%%%%%%%%%%%%%%%%%%
%%%%%%%%%%%%%%%%%%%%%%%%%%%%%%%%%%%%%%%%%%%%%%%%%%%%%%%%%%%%%%%%%%

%\pagebreak
\section*{Problem 1}
\subsection*{Solution}
\begin{enumerate}[(a)]
  \item Because $\forall i$, $f_i$ is postive non-decreasing, $\prod_{i=1}^{k'} f_i$, $\forall 1 \leq k' \leq k$ is also non-decreasing. Thus,
  \[
  \E{\prod_{i=1}^{k} f_i(\vect{X})} \leq \E{\prod_{i=1}^{k-1} f_i(\vect{X})} \E{f_k(\vect{X})} \leq \dots \leq \prod_{i=1}^{k}\E{f_i(\vect{X})}.
  \]
  \item NA r.v.s $\{X_1, \dots, X_n\}$ may not be independent, but if we can prove that Chernoff's trick also holds for NA r.v.s, then Hoeffding's inequality holds too. Using the results in Problem 2.(a).ii, $\{X_1 - \E{X_1}, \dots, X_n - \E{X_n}\}$ are also NA. For any $a > 0$ and $\theta > 0$, we have
  \[
    P(\sum_{i=1}^{n} X_i - \E{X_i} \geq a) \leq \frac{\E{\prod_{i=1}^{n} e^{\theta (X_i - \E{X_i})}}}{e^{\theta a}} \leq \frac{\prod_{i=1}^{n} \E{e^{\theta (X_i - \E{X_i})}}}{e^{\theta a}},
  \]
  where the first derivation uses Markov's inequality, and the last derivation uses the result in (a). Suppose $X_i \in [a_i, b_i]$, then we have
  \[
    P(\sum_{i=1}^{n} X_i - \E{X_i} \geq a) \leq \exp\left(\prod_{i=1}^{n} \frac{(b_i - a_i)^2}{8} \theta^2 - \theta a\right) \leq \exp\left(\frac{-2x^2}{\sum_{i=1}^{n} (b_i - a_i)^2}\right)
  \]
\end{enumerate}
%%%%%%%%%%%%%%%%%%%%%%%%%%%%%%%%%%%%%%%%%%%%%%%%%%%%%%%%%%%%%%%%%%
%%%%%%%%%%%%%%%%%%%%%%%%%%%%%%%%%%%%%%%%%%%%%%%%%%%%%%%%%%%%%%%%%%
%%%%%%%%%%%%%%%%%%%%%%%%%%%%%%%%%%%%%%%%%%%%%%%%%%%%%%%%%%%%%%%%%%
\section*{Problem 2}
\subsection*{Solution}
\begin{enumerate}[(a)]
  \item
  \begin{enumerate}[(i)]
    \item Here, we only consider increasing functions. Proof for decreasing functions is similar. If $f(X,\,Y)$ is an increasing function of $(X,\,Y)$, then given $Y=y$, $f(X,\,y)$ is an increasing function of $X$. Also, for any $y_1 \leq y_2$, we have $\forall X$, $f(X,\,y_1) \leq f(X,\,y_2)$ and thus $\E{f(X,\,y_1)} \leq \E{f(X,\,y_2)}$. Therefore, as a function of $Y$, $\E{\condi{f(X,\,Y)}{Y}}$ is also increasing. Thus, for any increasing function $f$ and $g$,
    \begin{multline*}
      \E{f(X,\,Y) \cdot g(X,\,Y)} = \E{\E{\condi{f(X,\,Y) \cdot g(X,\,Y)}{Y}}} \\\leq \E{\E{\condi{f(X,\,Y)}{Y}} \cdot \E{\condi{g(X,\,Y)}{Y}}} \leq \E{\E{\condi{f(X,\,Y)}{Y}}} \cdot \E{\E{\condi{g(X,\,Y)}{Y}}} \\= \E{f(X,\,Y)} \cdot \E{g(X,\,Y)}
    \end{multline*}
    \item For any two monotone functions $g$ and $h$ depending on disjoint subsets of $Y$, $g(Y) = g(\{Y_i,~i \in \mc I\}) = g(\{f_i(X),~i \in \mc I\})$ and $h(Y) = h(\{f_i(X),~i \in \mc J\})$. Since $\{f_i(X),~i \in [k]\}$ depends on disjoint subsets of $X$, $g$ and $h$ also depend on disjoint subsets of $X$. Also, since monotone functions are closed under composition, both $g$ and $h$ are still monotone (either increasing or decreasing) functions of $X$ depending disjoint subsets of $X$. Thus, $\E{g(Y) \cdot h(Y)} \leq \E{g(Y)} \cdot \E{h(Y)}$, and thus $Y^k$ are NA.
  \end{enumerate}
  \item Let $I_{j,i}$ be an indicator random variable to show if the $j$-th ball falls into the $i$-th bin. Then, for any $j \in [m]$, $\sum_{i=1}^{n} I_{j,i} = 1$. Next, we will show $I = \{I_{j,i}, j \in [m]\}$ are NA. For any increasing functions $f$ and $g$, denoting $\tilde{f}(I)$ and $\tilde{g}(I)$ as $f(I)-f(\vect{0})$ and $g(I)-g(\vect{0})$ respectively, then we have $\tilde{f}(I) \geq 0$ and $\tilde{g}(I) \geq 0$. Since only one element of $I$ can be 1, and $f$ and $g$ depend on disjoint subsets of $I$, at least one of $\tilde{f}(I)$ and $\tilde{g}(I)$ is zero, so $\E{\tilde{f}(I) \cdot \tilde{g}(I)} = 0$. Thus, $LHS = \E{f(I) \cdot g(I)} = \E{(\tilde{f}(I) + f(\vect{0})) \cdot (\tilde{g}(I) + g(\vect{0}))} = g(\vect{0}) \E{\tilde{f}(I)} + f(\vect{0})\E{\tilde{g}(I)} + f(\vect{0})g(\vect{0})$, and $RHS = (\E{\tilde{f}(I)}+ f(\vect{0})) (\E{\tilde{g}(I)}+ g(\vect{0})) \geq LHS$. Thus, $\{I_{j,i}, j \in [m]\}$ are NA.

  Using the results in (a.i), we have $\{I_{j,i},\text{ for any }j \in [m] \text{ and }i \in [n]\}$ are NA. Using the results in (a.ii), we have $X_i = \sum_{j=1}^{m} I_{j,i}$ are NA.
  \item Let $Y_i$ be an indicator random variable to indicate if the $i$-th bin is non-empty. We have $Y_i \sim Ber(1-(\frac{n-1}{n})^m)$, and $\E{Y_i} = 1-(\frac{n-1}{n})^m$. Using the results in (a.ii), $Y_i = \min\{1, X_i\}$ is NA. Using the results in Problem 1 (b), we have $\forall o > 0$,
  \begin{multline*}
    P\left(O-\E{O} = \sum_{i=1}^{n} (Y_i - \E{Y_i}) \geq o \right) \leq \exp\left(\sum_{i=1}^{n} \frac{(b_i-a_i)^2}{8}\theta^2 - \theta o \right) = \exp\left(\frac{n}{8} \theta^2 - \theta o \right)\\ \leq e^{\frac{-2 o^2}{n}}
  \end{multline*}
\end{enumerate}
%%%%%%%%%%%%%%%%%%%%%%%%%%%%%%%%%%%%%%%%%%%%%%%%%%%%%%%%%%%%%%%%%%
%%%%%%%%%%%%%%%%%%%%%%%%%%%%%%%%%%%%%%%%%%%%%%%%%%%%%%%%%%%%%%%%%%
%%%%%%%%%%%%%%%%%%%%%%%%%%%%%%%%%%%%%%%%%%%%%%%%%%%%%%%%%%%%%%%%%%
\section*{Problem 3}
\subsection*{Solution}
Let $Z_i \in \{1,2,3, \dots, n\}$ be the index of the bin into which the $i$-th ball is put. Let $f$ be a function such that $f(Z_1, Z_2, Z_3, \dots, Z_m)$ is the number of non-empty bins. Obviously, $Z_i$'s are independent, and $\left|f(Z_1, \dots, Z_i, \dots, Z_m) - f(Z_1, \dots, Z_i', \dots, Z_m)\right| < 1$. Thus, McDiarmid's inequality applys: $\forall o > 0$,
\[
  P(O-\E{O} = f(Z_1, Z_2, Z_3, \dots, Z_m) - \E{f(Z_1, Z_2, Z_3, \dots, Z_m)} \geq o) \leq e^\frac{-2 o^2}{n},
\]
which is the same as the Hoeffding bound.
%%%%%%%%%%%%%%%%%%%%%%%%%%%%%%%%%%%%%%%%%%%%%%%%%%%%%%%%%%%%%%%%%%
%%%%%%%%%%%%%%%%%%%%%%%%%%%%%%%%%%%%%%%%%%%%%%%%%%%%%%%%%%%%%%%%%%
%%%%%%%%%%%%%%%%%%%%%%%%%%%%%%%%%%%%%%%%%%%%%%%%%%%%%%%%%%%%%%%%%%
\section*{Problem 4}
\subsection*{Solution}
\begin{enumerate}[(a)]
  \item
  We have $\E{\exp(-\theta X)} \leq \exp(\nu^2\theta^2/2)$. Thus for any $t > 0$ and $\theta > 0$,
  \begin{multline*}
    P(|X| > t) = P(X > t) + P(-X > t) = P(e^{\theta X} > e^{\theta t}) + P(e^{-\theta X} > e^{\theta t}) \\\leq \frac{\E{e^{\theta X}}}{e^{\theta t}} + \frac{\E{e^{-\theta X}}}{e^{\theta t}} \leq 2 \exp(\nu^2\theta^2/2  - \theta t) \leq 2 \exp(-\frac{t^2}{2 \nu^2})
  \end{multline*}
  \item Let the CDF of $|X|$ be $F$. Then, $F(x) = 0$, $\forall x \leq 0$. Because $g(x) = x^k$ is an increasing function, using the Integration-by-parts Theorem \footnote{See Example 7 in \url{https://www.math.arizona.edu/~jwatkins/g-expectation.pdf}}, we have $$\E{|X|^k} = \int_{0}^{\infty} g'(x) P(|X|>x) dx \leq 2 \int_{0}^{\infty} kx^{k-1} \exp(-\frac{x^2}{2 \nu^2}) dx.$$ Subtitute $\frac{x^2}{2 \nu^2}$ with $y$, and we have $$\E{|X|^k} \leq 2 k \int_{0}^{\infty} \nu^2 (2 \nu^2 y)^{k/2-1} \exp(-y) dy = k (2\nu^2)^{k/2} \cdot \Gamma(\frac{k}{2}).$$
\end{enumerate}
%%%%%%%%%%%%%%%%%%%%%%%%%%%%%%%%%%%%%%%%%%%%%%%%%%%%%%%%%%%%%%%%%%
%%%%%%%%%%%%%%%%%%%%%%%%%%%%%%%%%%%%%%%%%%%%%%%%%%%%%%%%%%%%%%%%%%
%%%%%%%%%%%%%%%%%%%%%%%%%%%%%%%%%%%%%%%%%%%%%%%%%%%%%%%%%%%%%%%%%%
\section*{Problem 5}
\subsection*{Solution}
First, we partation the indices into $K$ index group such that $G_k = \{j: X_j \in A_k\}$. We also have for any $f \in F$ and $j_1,\,j_2 \in G_i$, $f(X_{j_1}) = f(X_{j_2})$. Thus,
\begin{multline*}
  R_n(F(X^n)) = \E{\sup\limits_{f \in F} \left|\frac{1}{n}\sum_{i=1}^{n}\sigma_i f(X_i)\right|} = \E{\sup\limits_{f \in F} \left|\frac{1}{n}\sum_{k=1}^{K} \sum_{i \in G_k} \sigma_i f(X_i)\right|} = \E{\frac{1}{n}\sum_{k=1}^{K} \left|\sum_{i \in G_k} \sigma_i\right|} \\\leq \E{\frac{1}{n}\sum_{k=1}^{K} \sqrt{\sum_{i \in G_k} \sigma_i^2}} = \E{\frac{1}{n}\sum_{k=1}^{K} \sqrt{n_k}} = \frac{1}{n}\sum_{k=1}^{K} \sqrt{n_k}
\end{multline*}
%%%%%%%%%%%%%%%%%%%%%%%%%%%%%%%%%%%%%%%%%%%%%%%%%%%%%%%%%%%%%%%%%%
%%%%%%%%%%%%%%%%%%%%%%%%%%%%%%%%%%%%%%%%%%%%%%%%%%%%%%%%%%%%%%%%%%
%%%%%%%%%%%%%%%%%%%%%%%%%%%%%%%%%%%%%%%%%%%%%%%%%%%%%%%%%%%%%%%%%%
\section*{Problem 6}
\subsection*{Solution}
For $n = 3$, we can choose $z_2 = z_1 + 0.55$ and $z_3 = z_2 + 0.55$. It is easy to verify $z_1$, $z_2$, and $z_3$ can be shattered by $F$. When $n = 4$, no matter what $z_1 < z_2 < z_3 < z_4$ are, they cannot be assigned values $1,\,0,\,1,\,0$. Thus $VC(F) = 3$.
%%%%%%%%%%%%%%%%%%%%%%%%%%%%%%%%%%%%%%%%%%%%%%%%%%%%%%%%%%%%%%%%%%
%%%%%%%%%%%%%%%%%%%%%%%%%%%%%%%%%%%%%%%%%%%%%%%%%%%%%%%%%%%%%%%%%%
%%%%%%%%%%%%%%%%%%%%%%%%%%%%%%%%%%%%%%%%%%%%%%%%%%%%%%%%%%%%%%%%%%
\section*{Problem 7}
\subsection*{Solution}
% First, $\bm 1_{\{\sin(\theta z > 0)\}} = \bm 1_{\{\floor*{\frac{\theta z}{\pi}} \text{ is even}\}}$
For $n$ points, we can choose $z_i = 2^i \pi$, $i=1,\,2,\,3,\,\dots,\,n$. Then $z^n$ can be shattered by $F$. Given a valuation $b_1,\,\dots,\,b_n$, we can choose $\theta = 2^{-n-1} + \sum_{i=1}^{n} 2^{-i} b_i$. Then, $\theta z_i = \pi (b_i + \sum_{j<i} 2^{i-j} b_j + \sum_{j>i}^{n} 2^{i-j} b_j + 2^{i-n-1})$, where $\sum_{j<i} 2^{i-j} b_j$ is an even number, and $\sum_{j>i}^{n} 2^{i-j} b_j + 2^{i-n-1}~\in~(0, 1)$. Thus, $\bm 1_{\{\sin(\theta z_i > 0)\}} = b_i$.
% Actually, we can find $2^n$ intervals $(1, \frac{2^n}{2^n-1})$, $(\frac{2^n}{2^n-1}, \frac{2^n}{2^n-2})$, $\dots$, $(\frac{2^n}{1}, \infty)$ such that when $\theta$ belongs to different intervals, $f$ assigns different values to $x^n$.
\end{document}
