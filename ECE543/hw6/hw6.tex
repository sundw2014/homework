\documentclass[11pt]{report}
% this template is originally from Roy Dong's ECE 515.
% Editted by Dawei Sun
%%%%%%%%%%%%%%%%%%%%%%%%%%%%%%%%%%%%%%%%%%%%%%%%%%%%%%%%%%%%%%%%%%
% Set the margins of our document.
\usepackage[margin = 1 in]{geometry}

%%%%%%%%%%%%%%%%%%%%%%%%%%%%%%%%%%%%%%%%%%%%%%%%%%%%%%%%%%%%%%%%%%
% Import commands for custom header.
\usepackage{fancyhdr}
\pagestyle{fancy}

%%%%%%%%%%%%%%%%%%%%%%%%%%%%%%%%%%%%%%%%%%%%%%%%%%%%%%%%%%%%%%%%%%
% Allow ourselves to do equations!
\usepackage{amsmath,amssymb,amsthm}
\usepackage{bm}
\usepackage{upgreek}

%%%%%%%%%%%%%%%%%%%%%%%%%%%%%%%%%%%%%%%%%%%%%%%%%%%%%%%%%%%%%%%%%%
% Nicer formatting for enumerate commands.
\usepackage[shortlabels]{enumitem}

%%%%%%%%%%%%%%%%%%%%%%%%%%%%%%%%%%%%%%%%%%%%%%%%%%%%%%%%%%%%%%%%%%
% Colored text and include images.
\usepackage{color}
\usepackage{graphicx}
\usepackage{mathtools}
\usepackage{bbm}

\usepackage{hyperref}
\hypersetup{
    colorlinks=true,
    linkcolor=blue,
    filecolor=magenta,
    urlcolor=cyan,
}

\urlstyle{same}

%%%%%%%%%%%%%%%%%%%%%%%%%%%%%%%%%%%%%%%%%%%%%%%%%%%%%%%%%%%%%%%%%%
% Some custom macros to make life easier.
\DeclarePairedDelimiter\ceil{\lceil}{\rceil}
\DeclarePairedDelimiter\floor{\lfloor}{\rfloor}
\newcommand{\mc}{\mathcal}
\newcommand{\mb}{\mathbb}
\newcommand{\vect}[1]{\boldsymbol{\mathbf{#1}}}
\newcommand{\T}{\intercal}
\newcommand{\E}[1]{\mathbb{E}\left[#1\right]}
\newcommand{\condi}[2]{#1 \ | \ #2}
\newcommand{\reals}{\mathbb{R}}
%%%%%%%%%%%%%%%%%%%%%%%%%%%%%%%%%%%%%%%%%%%%%%%%%%%%%%%%%%%%%%%%%%
%%%%%%%%%%%%%%%%%%%%%%%%%%%%%%%%%%%%%%%%%%%%%%%%%%%%%%%%%%%%%%%%%%
%%%%%%%%%%%%%%%%%%%%%%%%%%%%%%%%%%%%%%%%%%%%%%%%%%%%%%%%%%%%%%%%%%

\lhead{ECE 543 - Spring 2020 at University of Illinois at Urbana-Champaign}
\rhead{\textcolor{red}{HOMEWORK6}}
\lfoot{Submitted by: Dawei Sun (\textcolor{red}{\textit{daweis2}})}

%%%%%%%%%%%%%%%%%%%%%%%%%%%%%%%%%%%%%%%%%%%%%%%%%%%%%%%%%%%%%%%%%%
%%%%%%%%%%%%%%%%%%%%%%%%%%%%%%%%%%%%%%%%%%%%%%%%%%%%%%%%%%%%%%%%%%
%%%%%%%%%%%%%%%%%%%%%%%%%%%%%%%%%%%%%%%%%%%%%%%%%%%%%%%%%%%%%%%%%%

\begin{document}

%%%%%%%%%%%%%%%%%%%%%%%%%%%%%%%%%%%%%%%%%%%%%%%%%%%%%%%%%%%%%%%%%%
%%%%%%%%%%%%%%%%%%%%%%%%%%%%%%%%%%%%%%%%%%%%%%%%%%%%%%%%%%%%%%%%%%
%%%%%%%%%%%%%%%%%%%%%%%%%%%%%%%%%%%%%%%%%%%%%%%%%%%%%%%%%%%%%%%%%%

%\pagebreak
\section*{Problem 1}
\subsection*{Solution}
We denote the ReLU function by $\sigma$. The neural network is $\tilde{f}(x) = \sigma(a_2\sigma(a_1 x + b_1)+b_2)$, where $a_1 = -1/\epsilon$, $b_1 = 1$, $a_2 = -1$ and $b_2 = 1$.
\section*{Problem 2}
\subsection*{Solution}
We denote the ReLU function by $\sigma$. First, we construct a neural network $\tilde{f}(x)$ to mimic $\mathbb{I}_{x \geq 0}$ except in the region $x \in [-\epsilon, 0]$. Simillar to Problem 1, $\tilde{f}(x) = \sigma(a_2\sigma(a_1 x + b_1)+b_2)$, where $a_1 = -1/\epsilon$, $b_1 = 0$, $a_2 = -1$ and $b_2 = 1$.

\noindent Then, the final neural network is $4\sigma(\sum_{i=1}^{4} \tilde{f}(z_i) - 3.5)-1$, where $z_1 = 1 - (x_1 + x_2)$, $z_2 = 1 - (x_1 - x_2)$, $z_3 = 1 - (-x_1 + x_2)$, $z_4 = 1 - (-x_1 - x_2)$. This neural network mimics the groud truth except for $\{x |\ 1 \leq |x_1| + |x_2| \leq 1 + \epsilon\}$. For $x \in \{x |\ |x_1| + |x_2| < 1\}$, we must have $z_i > 0$ and $\tilde{f}(z_i) = 1$ for all $i$. Thus, the output is 1. For $x \in \{x |\ |x_1| + |x_2| > 1 + \epsilon \}$, at least one $z_i < -\epsilon$ and $\tilde{f}(z_i) = 0$. Thus, $\sum_{i=1}^{4} \tilde{f}(z_i) \leq 3$, and the output is $-1$.
\end{document}
