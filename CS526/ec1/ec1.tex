\documentclass[11pt]{report}
% this template is originally from Roy Dong's ECE 515.
% Editted by Dawei Sun
%%%%%%%%%%%%%%%%%%%%%%%%%%%%%%%%%%%%%%%%%%%%%%%%%%%%%%%%%%%%%%%%%%
% Set the margins of our document.
\usepackage[margin = 1 in]{geometry}

%%%%%%%%%%%%%%%%%%%%%%%%%%%%%%%%%%%%%%%%%%%%%%%%%%%%%%%%%%%%%%%%%%
% Import commands for custom header.
\usepackage{fancyhdr}
\pagestyle{fancy}

%%%%%%%%%%%%%%%%%%%%%%%%%%%%%%%%%%%%%%%%%%%%%%%%%%%%%%%%%%%%%%%%%%
% Allow ourselves to do equations!
\usepackage{amsmath,amssymb,amsthm}
\usepackage{bm}
\usepackage{upgreek}

%%%%%%%%%%%%%%%%%%%%%%%%%%%%%%%%%%%%%%%%%%%%%%%%%%%%%%%%%%%%%%%%%%
% Nicer formatting for enumerate commands.
\usepackage[shortlabels]{enumitem}

%%%%%%%%%%%%%%%%%%%%%%%%%%%%%%%%%%%%%%%%%%%%%%%%%%%%%%%%%%%%%%%%%%
% Colored text and include images.
\usepackage{color}
\usepackage{graphicx}
\usepackage{mathtools}
\usepackage{bbm}
\usepackage{diagbox}
\usepackage{hyperref}
\hypersetup{
    colorlinks=true,
    linkcolor=blue,
    filecolor=magenta,
    urlcolor=cyan,
}

\urlstyle{same}

%%%%%%%%%%%%%%%%%%%%%%%%%%%%%%%%%%%%%%%%%%%%%%%%%%%%%%%%%%%%%%%%%%
% Some custom macros to make life easier.
\DeclarePairedDelimiter\ceil{\lceil}{\rceil}
\DeclarePairedDelimiter\floor{\lfloor}{\rfloor}
\newcommand{\mc}{\mathcal}
\newcommand{\mb}{\mathbb}
\newcommand{\vect}[1]{\boldsymbol{\mathbf{#1}}}
\newcommand{\T}{\intercal}
\newcommand{\E}[1]{\mathbb{E}\left[#1\right]}
\newcommand{\condi}[2]{#1 \ | \ #2}
%%%%%%%%%%%%%%%%%%%%%%%%%%%%%%%%%%%%%%%%%%%%%%%%%%%%%%%%%%%%%%%%%%
%%%%%%%%%%%%%%%%%%%%%%%%%%%%%%%%%%%%%%%%%%%%%%%%%%%%%%%%%%%%%%%%%%
%%%%%%%%%%%%%%%%%%%%%%%%%%%%%%%%%%%%%%%%%%%%%%%%%%%%%%%%%%%%%%%%%%

\lhead{CS 526 - Spring 2020 at University of Illinois at Urbana-Champaign}
\rhead{\textcolor{red}{EC1}}
\lfoot{Submitted by: Dawei Sun (\textcolor{red}{\textit{daweis2}})}

%%%%%%%%%%%%%%%%%%%%%%%%%%%%%%%%%%%%%%%%%%%%%%%%%%%%%%%%%%%%%%%%%%
%%%%%%%%%%%%%%%%%%%%%%%%%%%%%%%%%%%%%%%%%%%%%%%%%%%%%%%%%%%%%%%%%%
%%%%%%%%%%%%%%%%%%%%%%%%%%%%%%%%%%%%%%%%%%%%%%%%%%%%%%%%%%%%%%%%%%

\begin{document}
\begin{enumerate}
  \item The lattice records a value for each variable. Suppose we have $n$ variables $v_1,\dots,v_n$ in the program. Then an example element of the lattice is $[v_1 \rightarrow \{+,0\}, \dots, v_n \rightarrow \{0,-\}]$.
  \item Forward analysis.
  \item $x_1 \vee x_2 = [x_1[v_1] \cup x_2[v_1], \dots, x_1[v_n] \cup x_2[v_n]]$.
  \item Here, $c$ is a constant. $f_{x=c}(V) = \begin{cases}
  V[x\rightarrow\{+\}]\text{ if }c\text{ is positive},\\
  V[x\rightarrow\{-\}]\text{ if }c\text{ is negative},\\
  V[x\rightarrow\{0\}]\text{ if }c=0\\\end{cases}.$
  \item Here, $c$ is a constant. The transfer function is shown as following

  \begin{tabular}{ c | c | c | c }
    \backslashbox{x}{c} & - & 0 & + \\\hline
    \{+\} & \{-,0,+\} & \{+\} & \{+\}\\\hline
    \{0,+\} & \{-,0,+\} & \{0,+\} & \{+\}\\\hline
    \{0\} & \{-\} & \{0\} & \{+\}\\\hline
    \{-,0\} & \{-\} & \{-,0\} & \{-,0,+\}\\\hline
    \{-\} & \{-\} & \{-\} & \{-,0,+\}\\\hline
    \{-,+\} & \{-,0,+\} & \{-,+\} & \{-,0,+\}\\\hline
    \{-,0,+\} & \{-,0,+\} & \{-,0,+\} & \{-,0,+\}\\\hline
    $\phi$ & $\phi$ & $\phi$ & $\phi$\\
  \end{tabular}
  \item $f_{x=c}(V)$ is distributive.
  \begin{multline*}
  f_{x=c}(V_1 \vee V_2) = f_{x=c}([V_1[x_1] \cup V_2[x_1],\dots,V_1[x] \cup V_2[x],\dots,V_1[x_n] \cup V_2[x_n]]) \\= [V_1[x_1] \cup V_2[x_1],\dots,sgn(c),\dots,V_1[x_n] \cup V_2[x_n]]
  \end{multline*}
  \begin{multline*}
  f_{x=c}(V_1) \vee f_{x=c}(V_2) \\= [V_1[x_1],\dots,sgn(c),\dots,V_1[x_n]] \vee [V_2[x_1],\dots,sgn(c),\dots,V_2[x_n]] \\= [V_1[x_1] \cup V_2[x_1],\dots,sgn(c),\dots,V_1[x_n] \cup V_2[x_n]] = f_{x=c}(V_1 \vee V_2)
  \end{multline*}

  $f_{x=x+c}(V)$ is distributive. By enumerating all the possible value of $V_1[x]$, $V_2[x]$, an $c$, we can prove that $f_{x=x+c}(V_1) \vee f_{x=x+c}(V_2) = f_{x=x+c}(V_1 \vee V_2)$ always holds.
  \item Yes. Since $F$ is distributive, it is monotone. Also, the lattice is finite, and thus the algorithm always terminates.
  \item Yes. According to [Kildall, 1973], since $F$ is distributive, MOP = MFP.
\end{enumerate}
\end{document}
